%% abtex2-modelo-glossarios.tex, v-1.9.6 laurocesar
%% Copyright 2012-2016 by abnTeX2 group at http://www.abntex.net.br/
%%
%% This work may be distributed and/or modified under the
%% conditions of the LaTeX Project Public License, either version 1.3
%% of this license or (at your option) any later version.
%% The latest version of this license is in
%%   http://www.latex-project.org/lppl.txt
%% and version 1.3 or later is part of all distributions of LaTeX
%% version 2005/12/01 or later.
%%
%% This work has the LPPL maintenance status `maintained'.
%%
%% The Current Maintainer of this work is the abnTeX2 team, led
%% by Lauro César Araujo. Further information are available on
%% http://www.abntex.net.br/
%%
%% This work consists of the files abtex2-modelo-glossarios.tex,
%% abntex2-modelo-include-comandos and abntex2-modelo-references.bib
%%
                     
% ------------------------------------------------------------------------
% ------------------------------------------------------------------------
% abnTeX2: Exemplo de glossários com o pacote glossaries e abntex2
% ------------------------------------------------------------------------
% ------------------------------------------------------------------------
 
\documentclass[
	% -- opções da classe memoir --
	12pt,				% tamanho da fonte
	openright,			% capítulos começam em pág ímpar (insere página vazia caso preciso)
	oneside,			% para impressão em recto e verso. Oposto a oneside
	a4paper,			% tamanho do papel. 
	% -- opções da classe abntex2 --
	%chapter=TITLE,		% títulos de capítulos convertidos em letras maiúsculas
	%section=TITLE,		% títulos de seções convertidos em letras maiúsculas
	%subsection=TITLE,	% títulos de subseções convertidos em letras maiúsculas
	%subsubsection=TITLE,% títulos de subsubseções convertidos em letras maiúsculas
	% -- opções do pacote babel --
	english,			% idioma adicional para hifenização
	french,				% idioma adicional para hifenização
	spanish,			% idioma adicional para hifenização
	brazil,				% o último idioma é o principal do documento
	]{abntex2}
 
 
% ---
% PACOTES
% ---
 
% ---
% Pacotes fundamentais
% ---
\usepackage{lmodern}			% Usa a fonte Latin Modern			
\usepackage[T1]{fontenc}		% Selecao de codigos de fonte.
\usepackage[utf8]{inputenc}		% Codificacao do documento (conversão automática dos acentos)
\usepackage{indentfirst}		% Indenta o primeiro parágrafo de cada seção.
\usepackage{color}				% Controle das cores
\usepackage{graphicx}			% Inclusão de gráficos
\usepackage{microtype} 			% para melhorias de justificação
% ---

% ---
% Pacotes glossaries
% ---
%\usepackage[subentrycounter,seeautonumberlist,nonumberlist=true]{glossaries}
% para usar o xindy ao invés do makeindex:
%\usepackage[xindy={language=portuguese},subentrycounter,seeautonumberlist,nonumberlist=true]{glossaries}
% ---

% ---
% Pacotes de citações
% ---
\usepackage[brazilian,hyperpageref]{backref}	 % Paginas com as citações na bibl
\usepackage[alf]{abntex2cite}	% Citações padrão ABNT
 
% ---
% Informações de dados para CAPA e FOLHA DE ROSTO
% ---
\titulo{Detalhamento dos Requisitos da Casa das Quentinhas}
\autor{César Antônio de Magalhães}
\local{Brasil}
\data{2017}
%\orientador{Lauro César Araujo}
%\coorientador{Equipe \abnTeX}
\tipotrabalho{Especificação de Caso de Uso}
% O preambulo deve conter o tipo do trabalho, o objetivo,
% o nome da instituição e a área de concentração
%\preambulo{Especificação de Caso de Uso.}
% ---
 
 
% ---
% Configurações de aparência do PDF final
 
% alterando o aspecto da cor azul
\definecolor{blue}{RGB}{41,5,195}
 
% informações do PDF
\makeatletter
\hypersetup{
     	%pagebackref=true,
		pdftitle={\@title}, 
		pdfauthor={\@author},
    	pdfsubject={\imprimirpreambulo},
	    pdfcreator={LaTeX with abnTeX2},
		pdfkeywords={abnt}{latex}{abntex}{abntex2}{glossários}, 
		colorlinks=true,       		% false: boxed links; true: colored links
    	linkcolor=blue,          	% color of internal links
    	citecolor=blue,        		% color of links to bibliography
    	filecolor=magenta,      		% color of file links
		urlcolor=blue,
		bookmarksdepth=4
}
\makeatother

% ---
 
% ---
% Espaçamentos entre linhas e parágrafos
% ---
 
% O tamanho do parágrafo é dado por:
\setlength{\parindent}{1.3cm}
 
% Controle do espaçamento entre um parágrafo e outro:
\setlength{\parskip}{0.2cm}  % tente também \onelineskip
 
% ---
% compila o indice
% ---
\makeindex
% ---
 
% ---
% GLOSSARIO
% ---
%\makeglossaries
 
% ---
% entradas do glossario
% ---
% \newglossaryentry{pai}{
%                name={pai},
%                plural={pai},
%                description={este é uma entrada pai, que possui outras
%                subentradas.} }

% \newglossaryentry{componente}{
%                name={componente},
%                plural={componentes},
%                parent=pai,
%                description={descriação da entrada componente.} }
 
% \newglossaryentry{filho}{
%                name={filho},
%                plural={filhos},
%                parent=pai,
%                description={isto é uma entrada filha da entrada de nome
%                \gls{pai}. Trata-se de uma entrada irmã da entrada
%                \gls{componente}.} }
 
%\newglossaryentry{equilibrio}{
%                name={equilíbrio da configuração},
%                see=[veja também]{componente},
%                description={consistência entre os \glspl{componente}}
%                }

%\newglossaryentry{latex}{
%                name={LaTeX},
%                description={ferramenta de computador para autoria de
%                documentos criada por D. E. Knuth} }

%\newglossaryentry{abntex2}{
%                name={abnTeX2},
%                see=latex,
%                description={suíte para LaTeX que atende os requisitos das
%                normas da ABNT para elaboração de documentos técnicos e científicos brasileiros} }
% ---

% ---
% Exemplo de configurações do glossairo
%\renewcommand*{\glsseeformat}[3][\seename]{\textit{#1}  
% \glsseelist{#2}}
% ---
              
                
% ----
% Início do documento
% ----
\begin{document}
 
% Retira espaço extra obsoleto entre as frases.
\frenchspacing
 
% ----------------------------------------------------------
% ELEMENTOS PRÉ-TEXTUAIS
% ----------------------------------------------------------
 
% ---
% Capa
% ---
\imprimircapa
% ---
 
% ---
% inserir o sumario
% ---
\pdfbookmark[0]{\contentsname}{toc}
\tableofcontents*
\cleardoublepage
% ---
 
% ----------------------------------------------------------
% ELEMENTOS TEXTUAIS
% ----------------------------------------------------------
\textual
 
% ----------------------------------------------------------
% Introdução
% ----------------------------------------------------------
\chapter*[Introdução]{Introdução}
\addcontentsline{toc}{chapter}{Introdução}

A empresa \textbf{Casa das Quentinhas}, realizou contato com a \textbf{Software House\&MM} e informou que necessita automatizar seu processo de entrega de marmitas. A Software House\&MM enviou um analista de sistemas para realizar o levantamento dos requisitos.

Durante a reunião, o proprietário da empresa Casa das Quentinhas, o Sr.Paulo Ricardo, informou que o seu negócio está em expansão, mas no momento a maior necessidade é automatizar as entregas e posteriormente os demais setores.

Com base no levantamento realizado pelo analista de sistemas, foi possível identificar os requisitos para o conjunto de funcionalidades referido como \textbf{Controle de Entrega de Marmitas}.
 
\chapter{Manter Cliente - UC001} \label{uc001}
 
\section{Breve descrição}
 
O sistema permitirá o cadastro e a alteração dos dados do cliente.

\section{Atores}

\begin{enumerate}
	\item Funcionário da empresa, Casa das Quentinhas.
\end{enumerate}

\section{Pré-condições}

\begin{enumerate}
	\item O funcionário deverá possuir login e senha de acesso autenticados pelo sistema.
	\item O funcionário deverá ter permissão para realizar o cadastro e a alteração dos dados do cliente.
\end{enumerate}

\section{Fluxo de eventos}

\subsection{Fluxo básico}

\begin{enumerate}[label=P\arabic*]
	\item O funcionário aciona a opção \textbf{Cliente >> Incluir} no menu do sistema. \label{uc001_p:1}\ref{uc001_a:1}
	\item O sistema apresenta a tela \textbf{Incluir Cliente} com os campos \ref{uc001_rn:1}. \label{uc001_p:2}
	\item O funcionário preenche os campos da tela. \label{uc001_p:3}
	\item O funcionário aciona a opção \textbf{Salvar}. \label{uc001_p:4}\ref{uc001_a:2}
	\item O sistema valida os dados dos campos. \ref{uc001_e:1} \ref{uc001_e:2} \ref{uc001_e:3} \ref{uc001_e:4} \ref{uc001_e:5} \ref{uc001_e:6} \ref{uc001_e:7}
	\item O sistema realiza a inclusão com sucesso.
	\item O sistema executa o caso de uso \nameref{uc013}.
	\item Esse caso de uso é encerrado.	
\end{enumerate}

\subsection{Fluxos alternativos}

\begin{enumerate}[label=A\arabic*]
	\item Alternativa ao passo \ref{uc001_p:1} - Alterar cliente \label{uc001_a:1}
	\begin{enumerate}[label*=.\arabic*]
		\item Na tela fornecida pelo sistema através do caso de uso \nameref{uc013}, o funcionário aciona a opção \textbf{Alterar}. 
		\item O sistema apresenta a tela \textbf{Alterar Cliente} com os campos \ref{uc001_rn:1}. \label{uc001_a:1:2}
		\item O funcionário preenche os campos da tela. \label{uc001_a:1:3}
		\item O funcionário aciona a opção \textbf{Salvar}. \label{uc001_a:1:4}\ref{uc001_a:2}
		\item O sistema valida os dados dos campos. \ref{uc001_e:1} \ref{uc001_e:2} \ref{uc001_e:3} \ref{uc001_e:4} \ref{uc001_e:5} \ref{uc001_e:6}
		\item O sistema altera os dados com sucesso.
		\item O sistema executa o caso de uso \nameref{uc013}.
		\item Esse caso de uso é encerrado.
	\end{enumerate}

	\item Alternativa ao passo \ref{uc001_p:4} ou \ref{uc001_a:1:4} - Cancelar inclusão ou alteração \label{uc001_a:2}
	\begin{enumerate}[label*=.\arabic*]
		\item O funcionário aciona a opção \textbf{Cancelar}.
		\item O sistema exibe a mensagem \textbf{Operação cancelada}.
		\item O funcionário aciona a opção \textbf{Ok}.
		\item O sistema retorna à tela principal.
		\item Esse caso de uso é encerrado.
	\end{enumerate}			 	
\end{enumerate}

\subsection{Exceções}

\begin{enumerate}[label=E\arabic*]
	\item O funcionário não informou algum campo obrigatório \label{uc001_e:1}
	\begin{enumerate}[label*=.\arabic*]
		\item[] No passo \ref{uc001_p:3} ou \ref{uc001_a:1:3}, o funcionário deixou em branco pelo menos um campo obrigatório.
		\item O sistema exibe a mensagem \textbf{Favor preencher o campo obrigatório}.
		\item O sistema destaca os campos não informados pelo funcionário.
		\item O sistema retorna ao passo anterior.
	\end{enumerate}

	\item O funcionário preencheu de forma errada o campo da data de nascimento \label{uc001_e:2}
	\begin{enumerate}[label*=.\arabic*]
			\item[] No passo \ref{uc001_p:3} ou \ref{uc001_a:1:3}, o funcionário não preencheu de forma correta o campo da data de nascimento, por exemplo, usou somente letras.				
		\item O sistema exibe a mensagem \textbf{A data de nascimento informada não é válida}.
		\item O sistema destaca o campo \textbf{Data de nascimento}.
		\item O sistema retorna ao passo anterior.
	\end{enumerate}

	\item O funcionário preencheu de forma errada o campo de telefone \label{uc001_e:3}
	\begin{enumerate}[label*=.\arabic*]		
		\item[] No passo \ref{uc001_p:3} ou \ref{uc001_a:1:3}, o funcionário não preencheu de forma correta o campo de telefone, usando letras ou qualquer outro carácter diferente de número.		
		\item O sistema exibe a mensagem \textbf{O número de telefone informado não é válido}.
		\item O sistema destaca o campo \textbf{Telefone}.
		\item O sistema retorna ao passo anterior.
	\end{enumerate}

	\item O funcionário preencheu de forma errada o campo de logradouro \label{uc001_e:4}
	\begin{enumerate}[label*=.\arabic*]		
		\item[] No passo \ref{uc001_p:3} ou \ref{uc001_a:1:3}, o funcionário não preencheu de forma correta o campo de logradouro (rua, avenida, beco, ...), por exemplo, usou somente números.		
		\item O sistema exibe a mensagem \textbf{O nome do logradouro informado não é válido}.
		\item O sistema destaca o campo \textbf{Logradouro}.
		\item O sistema retorna ao passo anterior.
	\end{enumerate}

	\item O funcionário preencheu de forma errada o campo de CEP \label{uc001_e:5}
	\begin{enumerate}[label*=.\arabic*]		
		\item[] No passo \ref{uc001_p:3} ou \ref{uc001_a:1:3}, o funcionário não preencheu de forma correta o campo de CEP, usando letras ou menos de 8 dígitos.		
		\item O sistema exibe a mensagem \textbf{O CEP informado não é válido}.
		\item O sistema destaca o campo \textbf{CEP}.
		\item O sistema retorna ao passo anterior.
	\end{enumerate}

	\item O funcionário preencheu de forma errada o campo de cidade \label{uc001_e:6}
	\begin{enumerate}[label*=.\arabic*]		
		\item[] No passo \ref{uc001_p:3} ou \ref{uc001_a:1:3}, o funcionário não preencheu de forma correta o campo de cidade, por exemplo, usou somente números.		
		\item O sistema exibe a mensagem \textbf{O nome da cidade informado não é válido}.
		\item O sistema destaca o campo \textbf{Cidade}.
		\item O sistema retorna ao passo anterior.
	\end{enumerate}

	\item Cliente cadastrado anteriormente \label{uc001_e:7}
	\begin{enumerate}[label*=.\arabic*]
		\item[] No passo \ref{uc001_p:3} do fluxo básico, o funcionário preencheu os dados de um cliente já cadastrado no sistema.
		\item O sistema exibe a mensagem \textbf{O cliente informado já foi cadastrado anteriormente no sistema}.
		\item O sistema retorna ao passo anterior.
	\end{enumerate}
\end{enumerate}

\section{Pós-condições}

\begin{enumerate}
	\item O funcionário terá cadastrado ou alterado os dados do cliente.
	\item O sistema executará o caso de uso \nameref{uc013}.	
\end{enumerate}

\section{Regras de negócios especiais}

\begin{enumerate}[label=RN\arabic*]
	\item Exibe os campos de dados do cliente de acordo com a tabela \ref{uc001_tb_rn1}. \label{uc001_rn:1}
	\begin{table}[htb]
		\ABNTEXfontereduzida
		\caption[Campos de dados do cliente]{Campos de dados do cliente.}
		\label{uc001_tb_rn1}
		\begin{tabular}{|p{3.0cm}|p{2.0cm}|p{1.5cm}|p{2.0cm}|p{5.75cm}|}
			\hline
			\textbf{Campo}      & \textbf{Tipo} & \textbf{Tamanho} & \textbf{Obrigatório} & \textbf{Observação}                                                                      \\ \hline
			Nome                & String        & 60               & SIM                  & N.A                                                                                      \\ \hline
			Data de nascimento  & Date          & N.A              & NÃO                  & O funcionário terá que informar a data de nascimento no formato \textbf{dia/mês/ano}.    \\ \hline
			Telefone            & String        & 10               & SIM                  & O funcionário terá que informar o número de telefone no formato \textbf{(99) 9999-9999}. \\ \hline
			Logradouro          & String        & 100              & SIM                  & N.A                                                                                      \\ \hline
			CEP                 & String        & 8                & SIM                  & O funcionário terá que informar o CEP no formato \textbf{99.999-999}.                    \\ \hline
			Bairro              & String        & 60               & SIM                  & N.A                                                                                      \\ \hline
			Cidade              & String        & 60               & SIM                  & N.A                                                                                      \\ \hline
			Número              & String        & 30               & SIM                  & N.A                                                                                      \\ \hline
			Complemento         & String        & 30               & NÃO                  & N.A                                                                                      \\ \hline
			Ponto de referência & String        & 30               & NÃO                  & N.A                                                                                      \\ \hline
		\end{tabular}
	\end{table}
\end{enumerate}
\chapter{Pesquisar Cliente - UC002} \label{uc002}

\section{Breve descrição}

O sistema exibirá os clientes que foram pesquisados pelo funcionário.

\section{Atores}

\begin{enumerate}
	\item Funcionário da empresa, Casa das Marmitas.
\end{enumerate}

\section{Pré-condições}

\begin{enumerate}
	\item O funcionário deverá possuir login e senha de acesso autenticados pelo sistema.
	\item O funcionário deverá ter permissão para realizar a consulta de clientes.
\end{enumerate}

\section{Fluxo de eventos}

\subsection{Fluxo básico}

\begin{enumerate}[label=P\arabic*]
	\item O funcionário aciona a opção \textbf{Cliente >> Listar} no menu do sistema.
	\item O sistema apresenta a tela \textbf{Lista de Clientes}.	
	\item O funcionário informa, no campo de pequisa, o número de telefone ou o nome do cliente. \label{uc002_p:3}
	\item O funcionário seleciona a opção \textbf{Buscar}. \label{uc002_p:4}
	\item O sistema exibe a lista de clientes com os campos \ref{uc002_rn:1}. \ref{uc002_e:1}
	\item O funcionário seleciona o cliente.
	\item O sistema executa o caso de uso \nameref{uc013}, de acordo com o cliente selecionado no passo anterior.
	\item Esse caso de uso é encerrado.
\end{enumerate}

\subsection{Exceções}

\begin{enumerate}[label=E\arabic*]
	\item Cliente não cadastrado \label{uc002_e:1}
	\begin{enumerate}[label*=.\arabic*]
		\item[] No passo \ref{uc002_p:3}, o funcionário informou o nome ou telefone de um cliente não cadastrado no sistema.
		\item O sistema exibe a mensagem \textbf{Cliente não cadastrado}.
		\item O sistema retorna ao passo anterior.
	\end{enumerate}
\end{enumerate}

\section{Pós-condições}

\begin{enumerate}
	\item O sistema executará o caso de uso \nameref{uc013}.	
\end{enumerate}

\section{Regras de negócios especiais}

\begin{enumerate}[label=RN\arabic*]
	\item Exibe os campos de dados do cliente de acordo com a tabela \ref{uc002_tb_rn1}. \label{uc002_rn:1}
	\begin{table}[htb]
		\ABNTEXfontereduzida
		\caption[Campos de dados do cliente]{Campos de dados do cliente.}
		\label{uc002_tb_rn1}
		\begin{tabular}{|p{4.0cm}|p{3.0cm}|p{7.25cm}|}
			\hline
			\textbf{Campo}      & \textbf{Tipo} & \textbf{Observação}                                                        \\ \hline
			Cód. do cliente     & Integer       & N.A                                                                        \\ \hline
			Data cadastro       & Date          & O sistema exibirá a data de cadastro no formato \textbf{dia/mês/ano}.      \\ \hline
			Nome                & String        & N.A                                                                        \\ \hline			
			Telefone            & String        & O sistema exibirá o número de telefone no formato \textbf{(99) 9999-9999}. \\ \hline			
		\end{tabular}
	\end{table}
\end{enumerate}
\chapter{Exibir Cliente - UC013} \label{uc013}

\section{Breve descrição}

Após a pesquisa ou o cadastro de um cliente, o sistema carregará um formulário com os dados do mesmo.

\section{Atores}

\begin{enumerate}
	\item Funcionário da empresa, Casa das Quentinhas.
\end{enumerate}

\section{Pré-condições}

\begin{enumerate}
	\item O funcionário deverá possuir login e senha de acesso autenticados pelo sistema.
	\item O funcionário deverá ter executado, anteriormente, o caso de uso \nameref{uc001} ou \nameref{uc002}.
\end{enumerate}

\section{Fluxo de eventos}

\subsection{Fluxo básico}

\begin{enumerate}[label=P\arabic*]
	\item O sistema apresenta a tela \textbf{Exibir Cliente} com os campos \ref{uc013_rn:1}. \label{uc013_p:1}
	\item O sistema exibe a lista de pedidos com os campos \ref{uc013_rn:2}. \label{uc013_p:2}\ref{uc013_e:1}
	\item O sistema habilita a opção \textbf{Alterar}. \label{uc013_p:3}\ref{uc013_a:1} 
	\item O sistema habilita a opção \textbf{Excluir}. \label{uc013_p:4}\ref{uc013_a:2}
	\item O sistema habilita opção \textbf{Registrar Pedido}. \label{uc013_p:5}\ref{uc013_a:3}
	\item O sistema habilita a seleção da lista de pedidos. \label{uc013_p:6}\ref{uc013_a:4}
	\item Esse caso de uso é encerrado. \label{uc013_p:7}\ref{uc013_a:5} \ref{uc013_a:6} \ref{uc013_a:7}  \ref{uc013_a:8}
\end{enumerate}

\subsection{Fluxos alternativos}

\begin{enumerate}[label=A\arabic*]
	\item Alternativa ao passo \ref{uc013_p:3} - O funcionário não tem permissão de alterar os dados do cliente \label{uc013_a:1}
	\begin{enumerate}[label*=.\arabic*]
		\item O sistema desabilita a opção \textbf{Alterar}.
		\item O sistema vai para o próximo passo.
	\end{enumerate}

	\item Alternativa ao passo \ref{uc013_p:4} - O funcionário não tem permissão de excluir o cliente \label{uc013_a:2}
	\begin{enumerate}[label*=.\arabic*]
		\item O sistema desabilita a opção \textbf{Excluir}.
		\item O sistema vai para o próximo passo.
	\end{enumerate}

	\item Alternativa ao passo \ref{uc013_p:5} - O funcionário não tem permissão de registrar pedidos para o cliente \label{uc013_a:3}
	\begin{enumerate}[label*=.\arabic*]
		\item O sistema desabilita a opção \textbf{Registrar Pedido}.
		\item O sistema vai para o próximo passo.
	\end{enumerate}
	
	\item Alternativa ao passo \ref{uc013_p:6} - O funcionário não tem permissão de alterar o status dos pedidos do cliente \label{uc013_a:4}
	\begin{enumerate}[label*=.\arabic*]
		\item O sistema desabilita a seleção da lista de pedidos.
		\item O sistema vai para o próximo passo.
	\end{enumerate}
	
	\item Alternativa ao passo \ref{uc013_p:7} - Alterar cliente \label{uc013_a:5}
	\begin{enumerate}[label*=.\arabic*]
		\item O funcionário aciona a opção \textbf{Alterar}, caso essa funcionalidade esteja habilitada.
		\item O sistema executa o caso de uso \nameref{uc001}.
		\item Esse caso de uso é encerrado.
	\end{enumerate}

	\item Alternativa ao passo \ref{uc013_p:7} - Excluir cliente \label{uc013_a:6}
	\begin{enumerate}[label*=.\arabic*]
		\item O funcionário aciona a opção \textbf{Excluir}, caso essa funcionalidade esteja habilitada.
		\item O sistema remove os dados do cliente. \label{uc013_a:6:2}\ref{uc013_e:2}
		\item O sistema volta à tela anterior.
		\item Esse caso de uso é encerrado.
	\end{enumerate}

	\item Alternativa ao passo \ref{uc013_p:7} - Registrar pedido \label{uc013_a:7}
	\begin{enumerate}[label*=.\arabic*]
		\item O funcionário aciona a opção \textbf{Registrar Pedido}, caso essa funcionalidade esteja habilitada.
		\item O sistema executa o caso de uso \nameref{uc007}.
		\item Esse caso de uso é suspenso.
	\end{enumerate}
	
	\item Alternativa ao passo \ref{uc013_p:7} - Alterar status do pedido \label{uc013_a:8}
	\begin{enumerate}[label*=.\arabic*]
		\item O funcionário seleciona o pedido, caso essa funcionalidade esteja habilitada.
		\item O sistema executa o caso de uso \nameref{uc009}.
		\item Esse caso de uso é suspenso.
	\end{enumerate}
\end{enumerate}

\subsection{Exceções}

\begin{enumerate}[label=E\arabic*]
	\item Cliente não tem pedido cadastrado \label{uc013_e:1}
	\begin{enumerate}[label*=.\arabic*]
		\item[] No passo \ref{uc013_p:2}, não foi encontrado nenhum pedido vinculado à conta do cliente.
		\item O sistema exibe a mensagem \textbf{O cliente não tem pedidos cadastrados}.
		\item O sistema vai para o próximo passo.
	\end{enumerate}

	\item Cliente tem pedido registrado \label{uc013_e:2}
	\begin{enumerate}[label*=.\arabic*]
		\item[] No passo \ref{uc013_a:6:2}, o funcionário tentou remover o cliente, mas isso não foi possível, por que esse tinha pelo menos um pedido cadastrado.
		\item O sistema exibe a mensagem \textbf{O cliente tem pedido registrado}.
		\item O sistema retorna ao passo \ref{uc013_p:1}.
	\end{enumerate}
\end{enumerate}

\section{Pós-condições}

\begin{enumerate}
	\item O sistema exibirá os dados do cliente.	
\end{enumerate}

\section{Regras de negócios especiais}

\begin{enumerate}[label=RN\arabic*]
	\item Exibe os campos de dados do cliente de acordo com a tabela \ref{uc013_tb_rn1}. \label{uc013_rn:1}
	\begin{table}[htb]
		\ABNTEXfontereduzida
		\caption[Campos de dados do cliente]{Campos de dados do cliente.}
		\label{uc013_tb_rn1}
		\begin{tabular}{|p{4.0cm}|p{3.0cm}|p{7.25cm}|}
			\hline
			\textbf{Campo}      & \textbf{Tipo} & \textbf{Observação}                                                        \\ \hline
			Cód. do cliente     & Integer       & N.A                                                                        \\ \hline
			Data cadastro       & Date          & O sistema exibirá a data de cadastro no formato \textbf{dia/mês/ano}.      \\ \hline
			Nome                & String        & N.A                                                                        \\ \hline
			Data de nascimento  & Date          & O sistema exibirá a data de nascimento no formato \textbf{dia/mês/ano}.    \\ \hline
			Telefone            & String        & O sistema exibirá o número de telefone no formato \textbf{(99) 9999-9999}. \\ \hline
			Logradouro          & String        & N.A                                                                        \\ \hline
			CEP                 & String        & O sistema exibirá o CEP no formato \textbf{99.999-999}.                    \\ \hline
			Bairro              & String        & N.A                                                                        \\ \hline
			Cidade              & String        & N.A                                                                        \\ \hline
			Número              & String        & N.A                                                                        \\ \hline
			Complemento         & String        & N.A                                                                        \\ \hline
			Ponto de referência & String        & N.A                                                                        \\ \hline
		\end{tabular}
	\end{table}
	
	\item Exibe os campos de dados do pedido de acordo com a tabela \ref{uc013_tb_rn2}. \label{uc013_rn:2}
	\begin{table}[htb]
		\ABNTEXfontereduzida
		\caption[Campos de dados do pedido]{Campos de dados do pedido.}
		\label{uc013_tb_rn2}
		\begin{tabular}{|p{4.0cm}|p{3.0cm}|p{7.25cm}|}
			\hline
			\textbf{Campo}   & \textbf{Tipo} & \textbf{Observação}                                                                                                                                              \\ \hline
			Cód. do pedido   & Long          & N.A                                                                                                                                                              \\ \hline
			Data cadastro    & Date          & O sistema exibirá a data de cadastro no formato \textbf{dia/mês/ano hora:minuto}.                                                                                \\ \hline
			Quantidade total & Short         & N.A                                                                                                                                                              \\ \hline
			Total do pedido  & Float         & O sistema exibirá, no formato moeda, o valor total do pedido.                                                                                                    \\ \hline
			Status           & Enum          & O sistema exibirá uma das seguintes opções: 
			\begin{enumerate}
				\item PENDENTE (pedido pendente);
				\item TRANSITO (pedido em trânsito);
				\item CANCELADO (pedido cancelado);
				\item ENTREGUE (pedido entregue).
			\end{enumerate}\\ \hline
		\end{tabular}
	\end{table}
\end{enumerate}

\chapter{Manter Entregador - UC003} \label{uc003}

\section{Breve descrição}

O serviço de entrega é terceirizado, com isso, os motoboys são vinculados a uma empresa e recebem apenas pelas entregas realizadas. Caso o motoboy não esteja cadastrado, o gerente irá cadastra-lo, informando os seus dados.

\section{Atores}

\begin{enumerate}
	\item Gerente da empresa, Casa das Marmitas.
\end{enumerate}

\section{Pré-condições}

\begin{enumerate}
	\item O gerente deverá possuir login e senha de acesso autenticados pelo sistema.
	\item O gerente deverá ter permissão para realizar o cadastro e a alteração dos dados do entregador.
\end{enumerate}

\section{Fluxo de eventos}

\subsection{Fluxo básico}

\begin{enumerate}[label=P\arabic*]
	\item O gerente aciona a opção \textbf{Entregador >> Incluir} no menu do sistema. \label{uc003_p:1}\ref{uc003_a:1}
	\item O sistema apresenta a tela \textbf{Incluir Entregador} com os campos \ref{uc003_rn:1}.
	\item No campo \textbf{Empresa}, o gerente informa o nome ou CNPJ da empresa. \label{uc003_p:3}
	\item O sistema carrega o caso de uso \nameref{uc011} de acordo com o dado informado no passo anterior.
	\item O gerente seleciona a empresa de entrega. \label{uc003_p:5}
	\item O gerente preenche os outros campos da tela. \label{uc003_p:6}	
	\item O gerente aciona a opção \textbf{Salvar}. \label{uc003_p:7}\ref{uc003_a:2}
	\item O sistema valida os dados dos campos. \ref{uc003_e:1} \ref{uc003_e:2} \ref{uc003_e:3} \ref{uc003_e:4} \ref{uc003_e:5}
	\item O sistema realiza a inclusão com sucesso.
	\item O sistema executa o caso de uso \nameref{uc016}.
	\item Esse caso de uso é encerrado.	
\end{enumerate}

\subsection{Fluxos alternativos}

\begin{enumerate}[label=A\arabic*]
	\item Alternativa ao passo \ref{uc003_p:1} - Alterar entregador \label{uc003_a:1}
	\begin{enumerate}[label*=.\arabic*]
		\item Na tela fornecida pelo sistema através do caso de uso \nameref{uc016}, o gerente aciona a opção \textbf{Alterar}. 
		\item O sistema apresenta a tela \textbf{Alterar Entregador} com os campos \ref{uc003_rn:1}. \label{uc003_a:1:2}
		\item O sistema desabilita o campo \textbf{Empresa}.
		\item O gerente preenche os campos da tela. \label{uc003_a:1:4}
		\item O gerente aciona a opção \textbf{Salvar}. \label{uc003_a:1:5}\ref{uc003_a:2}
		\item O sistema valida os dados dos campos. \ref{uc003_e:2} \ref{uc003_e:3} \ref{uc003_e:4}
		\item O sistema altera os dados com sucesso.
		\item O sistema executa o caso de uso \nameref{uc016}.
		\item Esse caso de uso é encerrado.
	\end{enumerate}
	
	\item Alternativa ao passo \ref{uc003_p:7} ou \ref{uc003_a:1:5} - Cancelar inclusão ou alteração \label{uc003_a:2}
	\begin{enumerate}[label*=.\arabic*]
		\item O gerente aciona a opção \textbf{Cancelar}.
		\item O sistema exibe a mensagem \textbf{Operação cancelada}.
		\item O gerente aciona a opção \textbf{Ok}.
		\item O sistema retorna à tela principal.
		\item Esse caso de uso é encerrado.
	\end{enumerate} 	
\end{enumerate}

\subsection{Exceções}

\begin{enumerate}[label=E\arabic*]
	\item O gerente não selecionou a empresa de entrega \label{uc003_e:1}
	\begin{enumerate}[label*=.\arabic*]
		\item[] No passo \ref{uc003_p:5}, o gerente não selecionou a empresa de entrega.
		\item O sistema exibe a mensagem \textbf{Informe a empresa de entrega}.
		\item O sistema destaca o campo \textbf{Empresa}.
		\item O sistema retorna ao passo \ref{uc003_p:3}.
	\end{enumerate}
	
	\item O gerente não informou algum campo obrigatório \label{uc003_e:2}
	\begin{enumerate}[label*=.\arabic*]
		\item[] No passo \ref{uc003_p:6} ou \ref{uc003_a:1:4}, o gerente deixou em branco pelo menos um campo obrigatório.
		\item O sistema exibe a mensagem \textbf{Favor preencher o campo obrigatório}.
		\item O sistema destaca os campos não informados pelo gerente.
		\item O sistema retorna ao passo anterior.
	\end{enumerate}

	\item O gerente preencheu de forma errada o campo de CPF \label{uc003_e:3}
	\begin{enumerate}[label*=.\arabic*]		
		\item[] No passo \ref{uc003_p:6} ou \ref{uc003_a:1:4}, o gerente não preencheu de forma correta o campo de CPF.		
		\item O sistema exibe a mensagem \textbf{CPF informado não é válido}.
		\item O sistema destaca o campo \textbf{CPF}.
		\item O sistema retorna ao passo anterior.
	\end{enumerate}

	\item O gerente preencheu de forma errada o campo de RG \label{uc003_e:4}
	\begin{enumerate}[label*=.\arabic*]		
		\item[] No passo \ref{uc003_p:6} ou \ref{uc003_a:1:4}, o gerente não preencheu de forma correta o campo de RG, por exemplo, usou letras ao invés de números, números excessivos ou menos números que o solicitado ou requerido.		
		\item O sistema exibe a mensagem \textbf{RG informado não é válido}.
		\item O sistema destaca o campo \textbf{RG}.
		\item O sistema retorna ao passo anterior.
	\end{enumerate}
		
	\item O gerente preencheu de forma errada o campo de celular \label{uc003_e:5}
	\begin{enumerate}[label*=.\arabic*]		
		\item[] No passo \ref{uc003_p:6} ou \ref{uc003_a:1:4}, o gerente não preencheu de forma correta o campo de celular, usando letras ou qualquer outro carácter diferente de número.		
		\item O sistema exibe a mensagem \textbf{O número de celular informado não é válido}.
		\item O sistema destaca o campo \textbf{Celular}.
		\item O sistema retorna ao passo anterior.
	\end{enumerate}	
	
	\item Entregador cadastrado anteriormente \label{uc003_e:6}
	\begin{enumerate}[label*=.\arabic*]
		\item[] No passo \ref{uc003_p:6} do fluxo básico, o gerente preencheu os dados de um entregador já cadastrado no sistema.
		\item O sistema exibe a mensagem \textbf{O entregador informado já foi cadastrado anteriormente no sistema}.
		\item O sistema retorna ao passo anterior.
	\end{enumerate}
\end{enumerate}

\section{Pós-condições}

\begin{enumerate}
	\item O gerente terá cadastrado ou alterado os dados do entregador.
	\item O sistema executará o caso de uso \nameref{uc016}.
\end{enumerate}

\section{Regras de negócios especiais}

\begin{enumerate}[label=RN\arabic*]
	\item Exibe os campos de dados do entregador de acordo com a tabela \ref{uc003_tb_rn1}. \label{uc003_rn:1}
	\begin{table}[htb]
		\ABNTEXfontereduzida
		\caption[Campos de dados do entregador]{Campos de dados do entregador.}
		\label{uc003_tb_rn1}
		\begin{tabular}{|p{3.0cm}|p{2.0cm}|p{1.5cm}|p{2.0cm}|p{5.75cm}|}
			\hline
			\textbf{Campo} & \textbf{Tipo} & \textbf{Tamanho} & \textbf{Obrigatório} & \textbf{Observação}                                                                  \\ \hline
			Empresa        & String        & 60               & SIM                  & N.A                                                                                  \\ \hline
			Nome           & String        & 60               & SIM                  & N.A                                                                                  \\ \hline
			CPF            & String        & 11               & SIM                  & O gerente terá que informar o CPF no formato \textbf{999.999.999-99}.                \\ \hline
			RG             & String        & 10               & SIM                  & O gerente terá que informar o RG no formato numérico.                                \\ \hline
			Celular        & String        & 11               & SIM                  & O gerente terá que informar o número de celular no formato \textbf{(99) 99999-9999}. \\ \hline
		\end{tabular}
	\end{table}
\end{enumerate}
\chapter{Pesquisar Entregador - UC015} \label{uc015}

\section{Breve descrição}

O sistema exibirá os entregadores que foram pesquisados pelo funcionário.

\section{Atores}

\begin{enumerate}
	\item Funcionário da empresa, Casa das Marmitas.
\end{enumerate}

\section{Pré-condições}

\begin{enumerate}
	\item O funcionário deverá possuir login e senha de acesso autenticados pelo sistema.
	\item O funcionário deverá ter permissão para realizar a consulta de entregadores.
\end{enumerate}

\section{Fluxo de eventos}

\subsection{Fluxo básico}

\begin{enumerate}[label=P\arabic*]
	\item O funcionário aciona a opção \textbf{Entregador >> Listar} no menu do sistema.
	\item O sistema apresenta a tela \textbf{Lista de Entregadores}.	
	\item O funcionário informa, no campo de pequisa, o nome, CPF, RG ou empresa do entregador. \label{uc015_p:3}
	\item O funcionário seleciona a opção \textbf{Buscar}.
	\item O sistema exibe a lista de entregadores com os campos \ref{uc015_rn:1}. \ref{uc015_e:1}
	\item O funcionário seleciona o entregador.
	\item O sistema executa o caso de uso \nameref{uc016}, de acordo com o entregador selecionado no passo anterior.
	\item Esse caso de uso é encerrado.
\end{enumerate}

\subsection{Exceções}

\begin{enumerate}[label=E\arabic*]
	\item Entregador não cadastrado \label{uc015_e:1}
	\begin{enumerate}[label*=.\arabic*]
		\item[] No passo \ref{uc015_p:3}, o funcionário informou o nome, CPF, RG ou empresa de um entregador não cadastrado no sistema.
		\item O sistema exibe a mensagem \textbf{Entregador não cadastrado}.
		\item O sistema retorna ao passo anterior.
	\end{enumerate}
\end{enumerate}

\section{Pós-condições}

\begin{enumerate}
	\item O sistema executará o caso de uso \nameref{uc016}.	
\end{enumerate}

\section{Regras de negócios especiais}

\begin{enumerate}[label=RN\arabic*]
	\item Exibe os campos de dados do entregador de acordo com a tabela \ref{uc015_tb_rn1}. \label{uc015_rn:1}
	\begin{table}[htb]
		\ABNTEXfontereduzida
		\caption[Campos de dados do entregador]{Campos de dados do entregador.}
		\label{uc015_tb_rn1}
		\begin{tabular}{|p{4.0cm}|p{3.0cm}|p{7.25cm}|}
			\hline
			\textbf{Campo}     & \textbf{Tipo} & \textbf{Observação}                                                        \\ \hline
			Cód. do entregador & Integer       & N.A                                                                        \\ \hline
			Empresa            & String        & N.A                                                                        \\ \hline
			Data cadastro      & Date          & O sistema exibirá a data de cadastro no formato \textbf{dia/mês/ano}.      \\ \hline
			Nome               & String        & N.A                                                                        \\ \hline
			Celular            & String        & O sistema exibirá o número de celular no formato \textbf{(99) 99999-9999}. \\ \hline
		\end{tabular}
	\end{table}
\end{enumerate}
\chapter{Visualiza Entregador - UC016} \label{uc016}

\section{Breve descrição}

Após a pesquisa ou o cadastro de um entregador, o sistema carregará um formulário com os dados do mesmo.

\section{Atores}

\begin{enumerate}
	\item Funcionário da empresa, Casa das Marmitas.
\end{enumerate}

\section{Pré-condições}

\begin{enumerate}
	\item O funcionário deverá possuir login e senha de acesso autenticados pelo sistema.
	\item O funcionário deverá ter executado, anteriormente, o caso de uso \nameref{uc003} ou \nameref{uc015}.
\end{enumerate}

\section{Fluxo de eventos}

\subsection{Fluxo básico}

\begin{enumerate}[label=P\arabic*]
	\item O sistema apresenta a tela \textbf{Exibir Entregador} com os campos \ref{uc016_rn:1}. \label{uc016_p:1}
	\item O sistema habilita a opção \textbf{Alterar}. \label{uc016_p:2}\ref{uc016_a:1} 
	\item O sistema habilita a opção \textbf{Excluir}. \label{uc016_p:3}\ref{uc016_a:2}
	\item Esse caso de uso é encerrado. \label{uc016_p:4}\ref{uc016_a:3} \ref{uc016_a:4}
\end{enumerate}

\subsection{Fluxos alternativos}

\begin{enumerate}[label=A\arabic*]
	\item Alternativa ao passo \ref{uc016_p:2} - O funcionário não tem permissão de alterar os dados do entregador \label{uc016_a:1}
	\begin{enumerate}[label*=.\arabic*]
		\item O sistema desabilita a opção \textbf{Alterar}.
		\item O sistema vai para o próximo passo.
	\end{enumerate}
	
	\item Alternativa ao passo \ref{uc016_p:3} - O funcionário não tem permissão de excluir o entregador \label{uc016_a:2}
	\begin{enumerate}[label*=.\arabic*]
		\item O sistema desabilita a opção \textbf{Excluir}.
		\item O sistema vai para o próximo passo.
	\end{enumerate}
		
	\item Alternativa ao passo \ref{uc016_p:4} - Alterar entregador \label{uc016_a:3}
	\begin{enumerate}[label*=.\arabic*]
		\item O funcionário aciona a opção \textbf{Alterar}, caso essa funcionalidade esteja habilitada.
		\item O sistema executa o caso de uso \nameref{uc003}.
		\item Esse caso de uso é encerrado.
	\end{enumerate}
	
	\item Alternativa ao passo \ref{uc016_p:4} - Excluir entregador \label{uc016_a:4}
	\begin{enumerate}[label*=.\arabic*]
		\item O funcionário aciona a opção \textbf{Excluir}, caso essa funcionalidade esteja habilitada.
		\item O sistema remove os dados do entregador. \label{uc016_a:4:2}
		\item O sistema volta à tela anterior.
		\item Esse caso de uso é encerrado.
	\end{enumerate}
\end{enumerate}

\section{Pós-condições}

\begin{enumerate}
	\item O sistema exibirá os dados do entregador.	
\end{enumerate}

\section{Regras de negócios especiais}

\begin{enumerate}[label=RN\arabic*]
	\item Exibe os campos de dados do entregador de acordo com a tabela \ref{uc016_tb_rn1}. \label{uc016_rn:1}
	\begin{table}[htb]
		\ABNTEXfontereduzida
		\caption[Campos de dados do entregador]{Campos de dados do entregador.}
		\label{uc016_tb_rn1}
		\begin{tabular}{|p{4.0cm}|p{3.0cm}|p{7.25cm}|}
			\hline
			\textbf{Campo}     & \textbf{Tipo} & \textbf{Observação}                                                        \\ \hline
			Cód. do entregador & Integer       & N.A                                                                        \\ \hline
			Empresa            & String        & N.A                                                                        \\ \hline
			Data cadastro      & Date          & O sistema exibirá a data de cadastro no formato \textbf{dia/mês/ano}.      \\ \hline
			Nome               & String        & N.A                                                                        \\ \hline
			CPF                & String        & O sistema exibirá o CPF no formato \textbf{999.999.999-99}.                \\ \hline
			RG                 & String        & N.A                                                                        \\ \hline
			Celular            & String        & O sistema exibirá o número de celular no formato \textbf{(99) 99999-9999}. \\ \hline
		\end{tabular}
	\end{table}
\end{enumerate}

\chapter{Manter Empresa - UC004} \label{uc004}

\section{Breve descrição}

Caso a empresa de entrega, onde os motoboys são vinculados, não esteja registrada, o gerente irá cadastra-la, informando os seus dados.

\section{Atores}

\begin{enumerate}
	\item Gerente da empresa, Casa das Marmitas.
\end{enumerate}

\section{Pré-condições}

\begin{enumerate}
	\item O gerente deverá possuir login e senha de acesso autenticados pelo sistema.
	\item O gerente deverá ter permissão para realizar o cadastro e a alteração dos dados da empresa.
\end{enumerate}

\section{Fluxo de eventos}

\subsection{Fluxo básico}

\begin{enumerate}[label=P\arabic*]
	\item O gerente aciona a opção \textbf{Empresa >> Incluir} no menu do sistema. \label{uc004_p:1}\ref{uc004_a:1}
	\item O sistema apresenta a tela \textbf{Incluir Empresa} com os campos \ref{uc004_rn:1}. \label{uc004_p:2}
	\item O gerente preenche os campos da tela. \label{uc004_p:3}
	\item O gerente aciona a opção \textbf{Salvar}. \label{uc004_p:4}\ref{uc004_a:2}
	\item O sistema valida os dados dos campos. \ref{uc004_e:1} \ref{uc004_e:2} \ref{uc004_e:3} \ref{uc004_e:4} \ref{uc004_e:5} \ref{uc004_e:6} \ref{uc004_e:7} \ref{uc004_e:8}
	\item O sistema realiza a inclusão com sucesso.
	\item O sistema executa o caso de uso \nameref{uc014}.
	\item Esse caso de uso é encerrado.	
\end{enumerate}

\subsection{Fluxos alternativos}

\begin{enumerate}[label=A\arabic*]
	\item Alternativa ao passo \ref{uc004_p:1} - Alterar empresa \label{uc004_a:1}
	\begin{enumerate}[label*=.\arabic*]
		\item Na tela fornecida pelo sistema através do caso de uso \nameref{uc014}, o gerente aciona a opção \textbf{Alterar}. 
		\item O sistema apresenta a tela \textbf{Alterar Empresa} com os campos \ref{uc004_rn:1}. \label{uc004_a:1:2}
		\item O gerente preenche os campos da tela. \label{uc004_a:1:3}
		\item O gerente aciona a opção \textbf{Salvar}. \label{uc004_a:1:4}\ref{uc004_a:2}
		\item O sistema valida os dados dos campos. \ref{uc004_e:1} \ref{uc004_e:2} \ref{uc004_e:3} \ref{uc004_e:4} \ref{uc004_e:5} \ref{uc004_e:6} \ref{uc004_e:7}
		\item O sistema altera os dados com sucesso.
		\item O sistema executa o caso de uso \nameref{uc014}.
		\item Esse caso de uso é encerrado.
	\end{enumerate}
	
	\item Alternativa ao passo \ref{uc004_p:4} ou \ref{uc004_a:1:4} - Cancelar inclusão ou alteração \label{uc004_a:2}
	\begin{enumerate}[label*=.\arabic*]
		\item O gerente aciona a opção \textbf{Cancelar}.
		\item O sistema exibe a mensagem \textbf{Operação cancelada}.
		\item O gerente aciona a opção \textbf{Ok}.
		\item O sistema retorna à tela principal.
		\item Esse caso de uso é encerrado.
	\end{enumerate}	 	
\end{enumerate}

\subsection{Exceções}

\begin{enumerate}[label=E\arabic*]	
	\item O gerente não informou algum campo obrigatório \label{uc004_e:1}
	\begin{enumerate}[label*=.\arabic*]
		\item[] No passo \ref{uc004_p:3} ou \ref{uc004_a:1:3}, o gerente deixou em branco pelo menos um campo obrigatório.
		\item O sistema exibe a mensagem \textbf{Favor preencher o campo obrigatório}.
		\item O sistema destaca os campos não informados pelo gerente.
		\item O sistema retorna ao passo anterior.
	\end{enumerate}
	
	\item O gerente preencheu de forma errada o campo de CNPJ \label{uc004_e:2}
	\begin{enumerate}[label*=.\arabic*]		
		\item[] No passo \ref{uc004_p:3} ou \ref{uc004_a:1:3}, o gerente não preencheu de forma correta o campo de CNPJ.		
		\item O sistema exibe a mensagem \textbf{CNPJ informado não é válido}.
		\item O sistema destaca o campo \textbf{CNPJ}.
		\item O sistema retorna ao passo anterior.
	\end{enumerate}
	
	\item O gerente preencheu de forma errada o campo de e-mail \label{uc004_e:3}
	\begin{enumerate}[label*=.\arabic*]		
		\item[] No passo \ref{uc004_p:3} ou \ref{uc004_a:1:3}, o gerente não preencheu de forma correta o campo de e-mail, por exemplo, usou somente letras ou colocou números excessivos	
		\item O sistema exibe a mensagem \textbf{E-mail informado não é válido}.
		\item O sistema destaca o campo \textbf{E-mail}.
		\item O sistema retorna ao passo anterior.
	\end{enumerate}
	
	\item O gerente preencheu de forma errada o campo de telefone \label{uc004_e:4}
	\begin{enumerate}[label*=.\arabic*]		
		\item[] No passo \ref{uc004_p:3} ou \ref{uc004_a:1:3}, o gerente não preencheu de forma correta o campo de telefone, usando letras ou qualquer outro carácter diferente de número.		
		\item O sistema exibe a mensagem \textbf{O número de telefone informado não é válido}.
		\item O sistema destaca o campo \textbf{Telefone}.
		\item O sistema retorna ao passo anterior.
	\end{enumerate}
	
	\item O gerente preencheu de forma errada o campo de logradouro \label{uc004_e:5}
	\begin{enumerate}[label*=.\arabic*]		
		\item[] No passo \ref{uc004_p:3} ou \ref{uc004_a:1:3}, o gerente não preencheu de forma correta o campo de logradouro (rua, avenida, beco, ...), por exemplo, usou somente números.		
		\item O sistema exibe a mensagem \textbf{O nome do logradouro informado não é válido}.
		\item O sistema destaca o campo \textbf{Logradouro}.
		\item O sistema retorna ao passo anterior.
	\end{enumerate}
	
	\item O gerente preencheu de forma errada o campo de CEP \label{uc004_e:6}
	\begin{enumerate}[label*=.\arabic*]		
		\item[] No passo \ref{uc004_p:3} ou \ref{uc004_a:1:3}, o gerente não preencheu de forma correta o campo de CEP, usando letras ou menos de 8 dígitos.		
		\item O sistema exibe a mensagem \textbf{O CEP informado não é válido}.
		\item O sistema destaca o campo \textbf{CEP}.
		\item O sistema retorna ao passo anterior.
	\end{enumerate}
	
	\item O gerente preencheu de forma errada o campo de cidade \label{uc004_e:7}
	\begin{enumerate}[label*=.\arabic*]		
		\item[] No passo \ref{uc004_p:3} ou \ref{uc004_a:1:3}, o gerente não preencheu de forma correta o campo de cidade, por exemplo, usou somente números.		
		\item O sistema exibe a mensagem \textbf{O nome da cidade informado não é válido}.
		\item O sistema destaca o campo \textbf{Cidade}.
		\item O sistema retorna ao passo anterior.
	\end{enumerate}	
	
	\item Empresa cadastrada anteriormente \label{uc004_e:8}
	\begin{enumerate}[label*=.\arabic*]
		\item[] No passo \ref{uc004_p:3} do fluxo básico, o gerente preencheu os dados de uma empresa já cadastrado no sistema.
		\item O sistema exibe a mensagem \textbf{A empresa informada já foi cadastrada anteriormente no sistema}.
		\item O sistema retorna ao passo anterior.
	\end{enumerate}
\end{enumerate}

\section{Pós-condições}

\begin{enumerate}
	\item O gerente terá cadastrado ou alterado os dados da empresa.
	\item O sistema executará o caso de uso \nameref{uc014}.	
\end{enumerate}

\section{Regras de negócios especiais}

\begin{enumerate}[label=RN\arabic*]
	\item Exibe os campos de dados da empresa de acordo com a tabela \ref{uc004_tb_rn1}. \label{uc004_rn:1}
	\begin{table}[htb]
		\ABNTEXfontereduzida
		\caption[Campos de dados da empresa]{Campos de dados da empresa.}
		\label{uc004_tb_rn1}
		\begin{tabular}{|p{3.0cm}|p{2.0cm}|p{1.5cm}|p{2.0cm}|p{5.75cm}|}
			\hline
			\textbf{Campo} & \textbf{Tipo} & \textbf{Tamanho} & \textbf{Obrigatório} & \textbf{Observação}                                                                  \\ \hline
			Nome           & String        & 60               & SIM                  & N.A                                                                                  \\ \hline
			CNPJ           & String        & 14               & SIM                  & O gerente terá que informar o CNPJ no formato \textbf{99.999.999/9999-99}.           \\ \hline
			E-mail         & String        & 30               & SIM                  & N.A                                                                                  \\ \hline
			Telefone       & String        & 10               & SIM                  & O gerente terá que informar o número de telefone no formato \textbf{(99) 9999-9999}. \\ \hline
			Logradouro     & String        & 100              & SIM                  & N.A                                                                                  \\ \hline
			CEP            & String        & 10               & SIM                  & O gerente terá que informar o CEP no formato \textbf{99.999-999}.                    \\ \hline
			Bairro         & String        & 60               & SIM                  & N.A                                                                                  \\ \hline
			Cidade         & String        & 60               & SIM                  & N.A                                                                                  \\ \hline
			Número         & String        & 30               & SIM                  & N.A                                                                                  \\ \hline
			Complemento    & String        & 30               & NÃO                  & N.A                                                                                  \\ \hline
		\end{tabular}
	\end{table}
\end{enumerate}
\chapter{Pesquisar Empresa - UC011} \label{uc011}

\section{Breve descrição}

O sistema exibirá as empresas que foram pesquisadas pelo funcionário.

\section{Atores}

\begin{enumerate}
	\item Funcionário da empresa, Casa das Quentinhas.
\end{enumerate}

\section{Pré-condições}

\begin{enumerate}
	\item O funcionário deverá possuir login e senha de acesso autenticados pelo sistema.
	\item O funcionário deverá ter permissão para realizar a consulta de empresas.
\end{enumerate}

\section{Fluxo de eventos}

\subsection{Fluxo básico}

\begin{enumerate}[label=P\arabic*]
	\item O funcionário aciona a opção \textbf{Empresa >> Listar} no menu do sistema. \label{uc011_p:1}\ref{uc011_a:1}
	\item O sistema apresenta a tela \textbf{Lista de Empresas}.	
	\item O funcionário informa, no campo de pequisa, o nome ou CNPJ da empresa. \label{uc011_p:3}
	\item O funcionário seleciona a opção \textbf{Buscar}.
	\item O sistema exibe a lista de empresas com os campos \ref{uc011_rn:1}. \ref{uc011_e:1}
	\item O funcionário seleciona a empresa.
	\item O sistema executa o caso de uso \nameref{uc014}, de acordo com a empresa selecionada no passo anterior.
	\item Esse caso de uso é encerrado.
\end{enumerate}

\subsection{Fluxos alternativos}

\begin{enumerate}[label=A\arabic*]
	\item Alternativa ao passo \ref{uc011_p:1} - Relacionar empresa ao entregador \label{uc011_a:1}
	\begin{enumerate}[label*=.\arabic*]
		\item Na tela fornecida pelo sistema através do caso de uso \nameref{uc003}, o funcionário preenche o campo \textbf{Empresa}. \label{uc011_a:1:1}
		\item O sistema exibe a lista de empresas, em uma tela, com os campos \ref{uc011_rn:1}. \ref{uc011_e:1}
		\item O funcionário seleciona a empresa. \label{uc011_a:1:3}
		\item O sistema atualiza o campo \textbf{Empresa} na tela \textbf{Incluir Entregador}.
		\item O sistema fecha a tela atual.
		\item O sistema retorna ao caso de uso \nameref{uc003}.
		\item Esse caso de uso é encerrado.
	\end{enumerate}
	
	\item Alternativa ao passo \ref{uc011_a:1:3} - Cancelar seleção
	\begin{enumerate}[label*=.\arabic*]
		\item O funcionário aciona a opção \textbf{Cancelar}.
		\item O sistema fecha a tela atual.
		\item O sistema retorna ao caso de uso \nameref{uc003}.
		\item Esse caso de uso é encerrado.
	\end{enumerate}
\end{enumerate}

\subsection{Exceções}

\begin{enumerate}[label=E\arabic*]
	\item Empresa não cadastrada \label{uc011_e:1}
	\begin{enumerate}[label*=.\arabic*]
		\item[] No passo \ref{uc011_p:3} ou \ref{uc011_a:1:1}, o funcionário informou o nome ou CNPJ de uma empresa não cadastrada no sistema.
		\item O sistema exibe a mensagem \textbf{Empresa não cadastrada}.
		\item O sistema retorna ao passo anterior.
	\end{enumerate}
\end{enumerate}

\section{Pós-condições}

\begin{itemize}
	\item Empresa selecionada na tela \textbf{Lista de Empresas}
	\begin{enumerate}
		\item O sistema executará o caso de uso \nameref{uc014}.	
	\end{enumerate}

	\item Empresa selecionada através da tela \textbf{Incluir Entregador}
	\begin{enumerate}
		\item O sistema carregará, no campo \textbf{Empresa}, o nome da empresa de entrega. 
	\end{enumerate}
\end{itemize}

\section{Regras de negócios especiais}

\begin{enumerate}[label=ED\arabic*]
	\item Exibe os campos de dados da empresa de acordo com a tabela \ref{uc011_tb_rn1}. \label{uc011_rn:1}
	\begin{table}[htb]
		\ABNTEXfontereduzida
		\caption[Campos de dados da empresa]{Campos de dados da empresa.}
		\label{uc011_tb_rn1}
		\begin{tabular}{|p{4.0cm}|p{3.0cm}|p{7.25cm}|}
			\hline
			\textbf{Campo}  & \textbf{Tipo} & \textbf{Observação}                                                        \\ \hline
			Cód. da empresa & Integer       & N.A                                                                        \\ \hline
			Data cadastro   & Date          & O sistema exibirá a data de cadastro no formato \textbf{dia/mês/ano}.      \\ \hline
			Nome            & String        & N.A                                                                        \\ \hline
			CNPJ            & String        & O sistema exibirá o CNPJ no formato \textbf{99.999.999/9999-99}.           \\ \hline
			E-mail          & String        & N.A                                                                        \\ \hline
			Telefone        & String        & O sistema exibirá o número de telefone no formato \textbf{(99) 9999-9999}. \\ \hline			
		\end{tabular}
	\end{table}
\end{enumerate}
\chapter{Exibir Empresa - UC014} \label{uc014}

\section{Breve descrição}

Após a pesquisa ou o cadastro de uma empresa, o sistema carregará um formulário com os dados da mesma.

\section{Atores}

\begin{enumerate}
	\item Funcionário da empresa, Casa das Quentinhas.
\end{enumerate}

\section{Pré-condições}

\begin{enumerate}
	\item O funcionário deverá possuir login e senha de acesso autenticados pelo sistema.
	\item O funcionário deverá ter executado, anteriormente, o caso de uso \nameref{uc004} ou \nameref{uc011}.
\end{enumerate}

\section{Fluxo de eventos}

\subsection{Fluxo básico}

\begin{enumerate}[label=P\arabic*]
	\item O sistema apresenta a tela \textbf{Exibir Empresa} com os campos \ref{uc014_rn:1}. \label{uc014_p:1}
	\item O sistema habilita a opção \textbf{Alterar}. \label{uc014_p:2}\ref{uc014_a:1} 
	\item O sistema habilita a opção \textbf{Excluir}. \label{uc014_p:3}\ref{uc014_a:2}
	\item Esse caso de uso é encerrado. \label{uc014_p:4}\ref{uc014_a:3} \ref{uc014_a:4}
\end{enumerate}

\subsection{Fluxos alternativos}

\begin{enumerate}[label=A\arabic*]
	\item Alternativa ao passo \ref{uc014_p:2} - O funcionário não tem permissão de alterar os dados da empresa \label{uc014_a:1}
	\begin{enumerate}[label*=.\arabic*]
		\item O sistema desabilita a opção \textbf{Alterar}.
		\item O sistema vai para o próximo passo.
	\end{enumerate}
	
	\item Alternativa ao passo \ref{uc014_p:3} - O funcionário não tem permissão de excluir a empresa \label{uc014_a:2}
	\begin{enumerate}[label*=.\arabic*]
		\item O sistema desabilita a opção \textbf{Excluir}.
		\item O sistema vai para o próximo passo.
	\end{enumerate}
	
	\item Alternativa ao passo \ref{uc014_p:4} - Alterar empresa \label{uc014_a:3}
	\begin{enumerate}[label*=.\arabic*]
		\item O funcionário aciona a opção \textbf{Alterar}, caso essa funcionalidade esteja habilitada.
		\item O sistema executa o caso de uso \nameref{uc004}.
		\item Esse caso de uso é encerrado.
	\end{enumerate}
	
	\item Alternativa ao passo \ref{uc014_p:4} - Excluir empresa \label{uc014_a:4}
	\begin{enumerate}[label*=.\arabic*]
		\item O funcionário aciona a opção \textbf{Excluir}, caso essa funcionalidade esteja habilitada.
		\item O sistema remove os dados da empresa. \label{uc014_a:4:2}\ref{uc014_e:1}
		\item O sistema volta à tela anterior.
		\item Esse caso de uso é encerrado.
	\end{enumerate}
\end{enumerate}

\subsection{Exceções}

\begin{enumerate}[label=E\arabic*]
	\item Empresa tem entregador vinculado \label{uc014_e:1}
	\begin{enumerate}[label*=.\arabic*]
		\item[] No passo \ref{uc014_a:4:2}, o funcionário tentou remover a empresa, mas isso não foi possível, por que essa tinha pelo menos um entregador vinculado a mesma.
		\item O sistema exibe a mensagem \textbf{A empresa tem entregador vinculado}.
		\item O sistema retorna ao passo \ref{uc014_p:1}.
	\end{enumerate}
\end{enumerate}

\section{Pós-condições}

\begin{enumerate}
	\item O sistema exibirá os dados da empresa.
\end{enumerate}

\section{Regras de negócios especiais}

\begin{enumerate}[label=ED\arabic*]
	\item Exibe os campos de dados da empresa de acordo com a tabela \ref{uc014_tb_rn1}. \label{uc014_rn:1}
	\begin{table}[htb]
		\ABNTEXfontereduzida
		\caption[Campos de dados da empresa]{Campos de dados da empresa.}
		\label{uc014_tb_rn1}
		\begin{tabular}{|p{4.0cm}|p{3.0cm}|p{7.25cm}|}
			\hline
			\textbf{Campo}  & \textbf{Tipo} & \textbf{Observação}                                                        \\ \hline
			Cód. da empresa & Integer       & N.A                                                                        \\ \hline
			Data cadastro   & Date          & O sistema exibirá a data de cadastro no formato \textbf{dia/mês/ano}.      \\ \hline
			Nome            & String        & N.A                                                                        \\ \hline
			CNPJ            & String        & O sistema exibirá o CNPJ no formato \textbf{99.999.999/9999-99}.           \\ \hline
			E-mail          & String        & N.A                                                                        \\ \hline
			Telefone        & String        & O sistema exibirá o número de telefone no formato \textbf{(99) 9999-9999}. \\ \hline
			Logradouro      & String        & N.A                                                                        \\ \hline
			CEP             & String        & O sistema exibirá o CEP no formato \textbf{99.999-999}.                    \\ \hline
			Bairro          & String        & N.A                                                                        \\ \hline
			Cidade          & String        & N.A                                                                        \\ \hline
			Número          & String        & N.A                                                                        \\ \hline
			Complemento     & String        & N.A                                                                        \\ \hline
		\end{tabular}
	\end{table}
\end{enumerate}

\chapter{Manter Produto - UC005} \label{uc005}

\section{Breve descrição}

O gerente cadastra as marmitas que serão vendidas, informando os seus dados.

\begin{table}[htb]
	\ABNTEXfontereduzida
	\caption[Dados referentes às marmitas]{Dados referentes às marmitas.}
	\centering
	\label{tab-marmitas}
	\begin{tabular}{|p{4cm}|p{8cm}|p{2cm}|}
		\hline
		\textbf{Nome do produto} & \textbf{Descrição}                               & \textbf{Custo} \\ \hline
		Marmita1                 & Arroz, feijão, bife e salada de tomate           & R\$ 15,00      \\ \hline
		Marmita2                 & Arroz, feijão, bife e ovo frito                  & R\$ 18,00      \\ \hline
		Marmita3                 & Arroz, feijão, file de frango, creme de milho    & R\$ 14,00      \\ \hline
		Marmita4                 & Arroz, feijão, file de frango e salada de tomate & R\$ 10,00      \\ \hline
	\end{tabular}
\end{table}

\section{Atores}

\begin{enumerate}
	\item Gerente da empresa, Casa das Quentinhas.
\end{enumerate}

\section{Pré-condições}

\begin{enumerate}
	\item O gerente deverá possuir login e senha de acesso autenticados pelo sistema.
	\item O gerente deverá ter permissão para realizar o cadastro e a alteração dos dados da marmita.
\end{enumerate}

\section{Fluxo de eventos}

\subsection{Fluxo básico}

\begin{enumerate}[label=P\arabic*]
	\item O gerente aciona a opção \textbf{Marmita >> Incluir} no menu do sistema. \label{uc005_p:1}\ref{uc005_a:1}
	\item O sistema apresenta a tela \textbf{Incluir Marmita} com os campos \ref{uc005_rn:1}. \label{uc005_p:2}
	\item O gerente preenche os campos da tela. \label{uc005_p:3}
	\item O gerente aciona a opção \textbf{Salvar}. \label{uc005_p:4}\ref{uc005_a:2}
	\item O sistema valida os dados dos campos. \ref{uc005_e:1} \ref{uc005_e:2} \ref{uc005_e:3}
	\item O sistema realiza a inclusão com sucesso.
	\item O sistema executa o caso de uso \nameref{uc017}.
	\item Esse caso de uso é encerrado.	
\end{enumerate}

\subsection{Fluxos alternativos}

\begin{enumerate}[label=A\arabic*]
	\item Alternativa ao passo \ref{uc005_p:1} - Alterar marmita \label{uc005_a:1}
	\begin{enumerate}[label*=.\arabic*]
		\item Na tela fornecida pelo sistema através do caso de uso \nameref{uc017}, o gerente aciona a opção \textbf{Alterar}. 
		\item O sistema apresenta a tela \textbf{Alterar Marmita} com os campos \ref{uc005_rn:1}. \label{uc005_a:1:2}
		\item O gerente preenche os campos da tela. \label{uc005_a:1:3}
		\item O gerente aciona a opção \textbf{Salvar}. \label{uc005_a:1:4}\ref{uc005_a:2}
		\item O sistema valida os dados dos campos. \ref{uc005_e:1} \ref{uc005_e:2}
		\item O sistema altera os dados com sucesso.
		\item O sistema executa o caso de uso \nameref{uc017}.
		\item Esse caso de uso é encerrado.
	\end{enumerate}
	
	\item Alternativa ao passo \ref{uc005_p:4} ou \ref{uc005_a:1:4} - Cancelar inclusão ou alteração \label{uc005_a:2}
	\begin{enumerate}[label*=.\arabic*]
		\item O gerente aciona a opção \textbf{Cancelar}.
		\item O sistema exibe a mensagem \textbf{Operação cancelada}.
		\item O gerente aciona a opção \textbf{Ok}.
		\item O sistema retorna à tela principal.
		\item Esse caso de uso é encerrado.
	\end{enumerate}	 	
\end{enumerate}

\subsection{Exceções}

\begin{enumerate}[label=E\arabic*]	
	\item O gerente não informou algum campo obrigatório \label{uc005_e:1}
	\begin{enumerate}[label*=.\arabic*]
		\item[] No passo \ref{uc005_p:3} ou \ref{uc005_a:1:3}, o gerente deixou em branco pelo menos um campo obrigatório.
		\item O sistema exibe a mensagem \textbf{Favor preencher o campo obrigatório}.
		\item O sistema destaca os campos não informados pelo gerente.
		\item O sistema retorna ao passo anterior.
	\end{enumerate}
	
	\item O gerente preencheu de forma errada o campo de custo \label{uc005_e:2}
	\begin{enumerate}[label*=.\arabic*]		
		\item[] No passo \ref{uc005_p:3} ou \ref{uc005_a:1:3}, o gerente não preencheu de forma correta o campo de custo.		
		\item O sistema exibe a mensagem \textbf{Custo informado não é válido}.
		\item O sistema destaca o campo \textbf{Custo}.
		\item O sistema retorna ao passo anterior.
	\end{enumerate}
		
	\item Marmita cadastrada anteriormente \label{uc005_e:3}
	\begin{enumerate}[label*=.\arabic*]
		\item[] No passo \ref{uc005_p:3} do fluxo básico, o gerente preencheu os dados de uma marmita já cadastrado no sistema.
		\item O sistema exibe a mensagem \textbf{A marmita informada já foi cadastrada anteriormente no sistema}.
		\item O sistema retorna ao passo anterior.
	\end{enumerate}
\end{enumerate}

\section{Pós-condições}

\begin{enumerate}
	\item O gerente terá cadastrado ou alterado os dados da marmita.
	\item O sistema executará o caso de uso \nameref{uc017}.	
\end{enumerate}

\section{Regras de negócios especiais}

\begin{enumerate}[label=RN\arabic*]
	\item Exibe os campos de dados da marmita de acordo com a tabela \ref{uc005_tb_rn1}. \label{uc005_rn:1}
	\begin{table}[htb]
		\ABNTEXfontereduzida
		\caption[Campos de dados da marmita]{Campos de dados da marmita.}
		\label{uc005_tb_rn1}
		\begin{tabular}{|p{3.0cm}|p{2.0cm}|p{1.5cm}|p{2.0cm}|p{5.75cm}|}
			\hline
			\textbf{Campo} & \textbf{Tipo} & \textbf{Tamanho} & \textbf{Obrigatório} & \textbf{Observação}                                              \\ \hline
			Nome           & String        & 60               & SIM                  & N.A                                                              \\ \hline
			Ingredientes   & String        & 250              & SIM                  & N.A                                                              \\ \hline
			Custo          & Float         & N.A              & SIM                  & O gerente terá que informar o custo da marmita no formato moeda. \\ \hline
			Tamanho        & Enum          & N.A              & SIM                  & O gerente terá que informar uma das seguintes opções: 	
			\begin{enumerate}
				\item GRANDE (tamanho grande);
				\item MEDIO (tamanho médio);
				\item PEQUENO (tamanho pequeno).
			\end{enumerate}\\ \hline
		\end{tabular}
	\end{table}
\end{enumerate}
\chapter{Pesquisa Produto - UC012} \label{uc012}

\section{Breve descrição}

O sistema exibirá as marmitas que foram pesquisadas pelo funcionário.

\section{Atores}

\begin{enumerate}
	\item Funcionário da empresa, Casa das Marmitas.
\end{enumerate}

\section{Pré-condições}

\begin{enumerate}
	\item O funcionário deverá possuir login e senha de acesso autenticados pelo sistema.
	\item O funcionário deverá ter permissão para realizar a busca de marmitas.
\end{enumerate}

\section{Fluxo de eventos}

\subsection{Fluxo básico}

\begin{enumerate}[label=P\arabic*]
	\item O funcionário aciona a opção \textbf{Marmita >> Listar} no menu do sistema. \label{uc012_p:1}\ref{uc012_a:1}
	\item O sistema apresenta a tela \textbf{Lista de Marmitas}.	
	\item O funcionário informa, no campo de pequisa, o código ou nome da marmita. \label{uc012_p:3}
	\item O funcionário seleciona a opção \textbf{Buscar}.
	\item O sistema exibe a lista de marmitas com os campos \ref{uc012_rn:1}. \ref{uc012_e:1}
	\item O funcionário seleciona a marmita.
	\item O sistema executa o caso de uso \nameref{uc017}, de acordo com a marmita selecionada no passo anterior.
	\item Esse caso de uso é encerrado.
\end{enumerate}

\subsection{Fluxos alternativos}

\begin{enumerate}[label=A\arabic*]
	\item Alternativa ao passo \ref{uc012_p:1} - Relacionar marmita ao pedido \label{uc012_a:1}
	\begin{enumerate}[label*=.\arabic*]
		\item Na tela fornecida pelo sistema através do caso de uso \nameref{uc007}, o funcionário preenche o campo \textbf{Produto}. \label{uc012_a:1:1}
		\item O sistema exibe a lista de marmitas, em uma tela, com os campos \ref{uc012_rn:1}. \ref{uc012_e:1}
		\item O funcionário seleciona a marmita. \label{uc012_a:1:3}
		\item O sistema atualiza o campo \textbf{Produto} na tela \textbf{Registrar Pedido}.
		\item O sistema fecha a tela atual.
		\item O sistema retorna ao caso de uso \nameref{uc007}.
		\item Esse caso de uso é encerrado.
	\end{enumerate}
	
	\item Alternativa ao passo \ref{uc012_a:1:3} - Cancelar seleção
	\begin{enumerate}[label*=.\arabic*]
		\item O funcionário aciona a opção \textbf{Cancelar}.
		\item O sistema fecha a tela atual.
		\item O sistema retorna ao caso de uso \nameref{uc007}.
		\item Esse caso de uso é encerrado.
	\end{enumerate}
\end{enumerate}

\subsection{Exceções}

\begin{enumerate}[label=E\arabic*]
	\item Marmita não cadastrada \label{uc012_e:1}
	\begin{enumerate}[label*=.\arabic*]
		\item[] No passo \ref{uc012_p:3} ou \ref{uc012_a:1:1}, o funcionário informou o código ou nome de uma marmita não cadastrada no sistema.
		\item O sistema exibe a mensagem \textbf{Marmita não cadastrada}.
		\item O sistema retorna ao passo anterior.
	\end{enumerate}
\end{enumerate}

\section{Pós-condições}

\begin{itemize}
	\item Marmita selecionada na tela \textbf{Lista de Marmitas}
	\begin{enumerate}
		\item O sistema executará o caso de uso \nameref{uc017}.	
	\end{enumerate}

	\item Marmita selecionada através da tela \textbf{Registrar Pedido}
	\begin{enumerate}
		\item O sistema carregará, no campo \textbf{Produto}, o nome da marmita. 
	\end{enumerate}
\end{itemize}

\section{Regras de negócios especiais}

\begin{enumerate}[label=ED\arabic*]
	\item Exibe os campos de dados da marmita de acordo com a tabela \ref{uc012_tb_rn1}. \label{uc012_rn:1}
	\begin{table}[htb]
		\ABNTEXfontereduzida
		\caption[Campos de dados da marmita]{Campos de dados da marmita.}
		\label{uc012_tb_rn1}
		\begin{tabular}{|p{4.0cm}|p{3.0cm}|p{7.25cm}|}
			\hline
			\textbf{Campo}  & \textbf{Tipo} & \textbf{Observação}                                                   \\ \hline
			Cód. do produto & Integer       & N.A                                                                   \\ \hline
			Data cadastro   & Date          & O sistema exibirá a data de cadastro no formato \textbf{dia/mês/ano}. \\ \hline
			Nome            & String        & N.A                                                                   \\ \hline			
			Custo           & Float         & O sistema exibirá o custo da marmita no formato moeda.                \\ \hline
			Tamanho           & Enum          & O sistema exibirá uma das seguintes opções: 	
			\begin{enumerate}
				\item GRANDE (tamanho grande);
				\item MEDIO (tamanho médio);
				\item PEQUENO (tamanho pequeno).
			\end{enumerate}\\ \hline
		\end{tabular}
	\end{table}
\end{enumerate}
\chapter{Exibir Produto - UC017} \label{uc017}

\section{Breve descrição}

Após a pesquisa ou o cadastro de uma marmita, o sistema carregará um formulário com os dados da mesma.

\section{Atores}

\begin{enumerate}
	\item Funcionário da empresa, Casa das Marmitas.
\end{enumerate}

\section{Pré-condições}

\begin{enumerate}
	\item O funcionário deverá possuir login e senha de acesso autenticados pelo sistema.
	\item O funcionário deverá ter executado, anteriormente, o caso de uso \nameref{uc005} ou \nameref{uc012}.
\end{enumerate}

\section{Fluxo de eventos}

\subsection{Fluxo básico}

\begin{enumerate}[label=P\arabic*]
	\item O sistema apresenta a tela \textbf{Exibir Marmita} com os campos \ref{uc017_rn:1}. \label{uc017_p:1}
	\item O sistema habilita a opção \textbf{Alterar}. \label{uc017_p:2}\ref{uc017_a:1} 
	\item O sistema habilita a opção \textbf{Excluir}. \label{uc017_p:3}\ref{uc017_a:2}
	\item Esse caso de uso é encerrado. \label{uc017_p:4}\ref{uc017_a:3} \ref{uc017_a:4}
\end{enumerate}

\subsection{Fluxos alternativos}

\begin{enumerate}[label=A\arabic*]
	\item Alternativa ao passo \ref{uc017_p:2} - O funcionário não tem permissão de alterar os dados da marmita \label{uc017_a:1}
	\begin{enumerate}[label*=.\arabic*]
		\item O sistema desabilita a opção \textbf{Alterar}.
		\item O sistema vai para o próximo passo.
	\end{enumerate}
	
	\item Alternativa ao passo \ref{uc017_p:3} - O funcionário não tem permissão de excluir a marmita \label{uc017_a:2}
	\begin{enumerate}[label*=.\arabic*]
		\item O sistema desabilita a opção \textbf{Excluir}.
		\item O sistema vai para o próximo passo.
	\end{enumerate}
	
	\item Alternativa ao passo \ref{uc017_p:4} - Alterar marmita \label{uc017_a:3}
	\begin{enumerate}[label*=.\arabic*]
		\item O funcionário aciona a opção \textbf{Alterar}, caso essa funcionalidade esteja habilitada.
		\item O sistema executa o caso de uso \nameref{uc005}.
		\item Esse caso de uso é encerrado.
	\end{enumerate}
	
	\item Alternativa ao passo \ref{uc017_p:4} - Excluir marmita \label{uc017_a:4}
	\begin{enumerate}[label*=.\arabic*]
		\item O funcionário aciona a opção \textbf{Excluir}, caso essa funcionalidade esteja habilitada.
		\item O sistema remove os dados da marmita. \label{uc017_a:4:2}\ref{uc017_e:1}
		\item O sistema volta à tela anterior.
		\item Esse caso de uso é encerrado.
	\end{enumerate}
\end{enumerate}

\subsection{Exceções}

\begin{enumerate}[label=E\arabic*]
	\item Marmita tem pedido vinculado \label{uc017_e:1}
	\begin{enumerate}[label*=.\arabic*]
		\item[] No passo \ref{uc017_a:4:2}, o funcionário tentou remover a marmita, mas isso não foi possível, por que essa tinha pelo menos um pedido vinculado a mesma.
		\item O sistema exibe a mensagem \textbf{A marmita tem pedido vinculado}.
		\item O sistema retorna ao passo \ref{uc017_p:1}.
	\end{enumerate}
\end{enumerate}

\section{Pós-condições}

\begin{enumerate}
	\item O sistema exibirá os dados da marmita.	
\end{enumerate}

\section{Regras de negócios especiais}

\begin{enumerate}[label=ED\arabic*]
	\item Exibe os campos de dados da marmita de acordo com a tabela \ref{uc017_tb_rn1}. \label{uc017_rn:1}
	\begin{table}[htb]
		\ABNTEXfontereduzida
		\caption[Campos de dados da marmita]{Campos de dados da marmita.}
		\label{uc017_tb_rn1}
		\begin{tabular}{|p{4.0cm}|p{3.0cm}|p{7.25cm}|}
			\hline
			\textbf{Campo}  & \textbf{Tipo} & \textbf{Observação}                                                   \\ \hline
			Cód. do produto & Integer       & N.A                                                                   \\ \hline
			Data cadastro   & Date          & O sistema exibirá a data de cadastro no formato \textbf{dia/mês/ano}. \\ \hline
			Nome            & String        & N.A                                                                   \\ \hline
			Ingrediente     & String        & N.A                                                                   \\ \hline
			Custo           & Float         & O sistema exibirá o custo da marmita no formato moeda.                \\ \hline
			Tamanho           & Enum          & O sistema exibirá uma das seguintes opções: 	
			\begin{enumerate}
				\item GRANDE (tamanho grande);
				\item MEDIO (tamanho médio);
				\item PEQUENO (tamanho pequeno).
			\end{enumerate}\\ \hline
		\end{tabular}
	\end{table}
\end{enumerate}

\chapter{Alterar Taxa de Entrega - UC006} \label{uc006}

\section{Breve descrição}

O gerente altera o valor da taxa de entrega, por exemplo:

\begin{itemize}
	\item Taxa de entrega: R\$ 4,50
\end{itemize}

\section{Atores}

\begin{enumerate}
	\item Gerente da empresa, Casa das Marmitas.
\end{enumerate}

\section{Pré-condições}

\begin{enumerate}
	\item O gerente deverá possuir login e senha de acesso autenticados pelo sistema.
	\item O gerente deverá ter permissão para realizar a alteração da taxa de entrega.
\end{enumerate}

\section{Fluxo de eventos}

\subsection{Fluxo básico}

\begin{enumerate}[label=P\arabic*]
	\item O gerente aciona a opção \textbf{Taxa de Entrega >> Alterar} no menu do sistema.
	\item O sistema apresenta a tela \textbf{Alterar Taxa de Entrega} com os campos \ref{uc006_rn:1}. \label{uc006_p:2}
	\item O gerente preenche os campos da tela. \label{uc006_p:3}
	\item O gerente aciona a opção \textbf{Salvar}. \label{uc006_p:4}\ref{uc006_a:2}
	\item O sistema valida os dados dos campos. \ref{uc006_e:1} \ref{uc006_e:2}
	\item O sistema altera os dados com sucesso.
	\item O sistema carrega a tela principal.
	\item Esse caso de uso é encerrado.	
\end{enumerate}

\subsection{Fluxos alternativos}

\begin{enumerate}[label=A\arabic*]
	\item Alternativa ao passo \ref{uc006_p:4} - Cancelar alteração \label{uc006_a:2}
	\begin{enumerate}[label*=.\arabic*]
		\item O gerente aciona a opção \textbf{Cancelar}.
		\item O sistema exibe a mensagem \textbf{Operação cancelada}.
		\item O gerente aciona a opção \textbf{Ok}.
		\item O sistema retorna à tela principal.
		\item Esse caso de uso é encerrado.
	\end{enumerate}	
\end{enumerate}

\subsection{Exceções}

\begin{enumerate}[label=E\arabic*]	
	\item O gerente não informou algum campo obrigatório \label{uc006_e:1}
	\begin{enumerate}[label*=.\arabic*]
		\item[] No passo \ref{uc006_p:3}, o gerente deixou em branco pelo menos um campo obrigatório.
		\item O sistema exibe a mensagem \textbf{Favor preencher o campo obrigatório}.
		\item O sistema destaca os campos não informados pelo gerente.
		\item O sistema retorna ao passo anterior.
	\end{enumerate}
	
	\item O gerente preencheu de forma errada o campo de taxa de entrega \label{uc006_e:2}
	\begin{enumerate}[label*=.\arabic*]		
		\item[] No passo \ref{uc006_p:3}, o gerente não preencheu de forma correta o campo de taxa de entrega.		
		\item O sistema exibe a mensagem \textbf{Taxa de entrega informada não é válida}.
		\item O sistema destaca o campo \textbf{Taxa de entrega}.
		\item O sistema retorna ao passo anterior.
	\end{enumerate}
\end{enumerate}

\section{Pós-condições}

\begin{enumerate}
	\item O gerente terá alterado a taxa de entrega.
	\item O sistema carregará a tela principal.	
\end{enumerate}

\section{Regras de negócios especiais}

\begin{enumerate}[label=RN\arabic*]
	\item Exibe os campos de dados da taxa de entrega de acordo com a tabela \ref{uc006_tb_rn1}. \label{uc006_rn:1}
	\begin{table}[htb]
		\ABNTEXfontereduzida
		\caption[Campos de dados da taxa de entrega]{Campos de dados da taxa de entrega.}
		\label{uc006_tb_rn1}
		\begin{tabular}{|p{3.0cm}|p{2.0cm}|p{1.5cm}|p{2.0cm}|p{5.75cm}|}
			\hline
			\textbf{Campo}  & \textbf{Tipo} & \textbf{Tamanho} & \textbf{Obrigatório} & \textbf{Observação}                                             \\ \hline
			Taxa de entrega & Float         & N.A              & SIM                  & O gerente terá que informar a taxa de entrega no formato moeda. \\ \hline
			Tipo            & Enum          & N.A              & SIM                  & O sistema exibirá a opção \textbf{ENTREGA}.                     \\ \hline
		\end{tabular}
	\end{table}
\end{enumerate}
\chapter{Registrar Pedido - UC007} \label{uc007}

\section{Breve descrição}

Para gerenciar os pedidos realizados pelo cliente, o atendente deverá informar os dados do pedido.

O sistema deverá calcular automaticamente o valor total do pedido, por exemplo:

\begin{itemize}
	\item Pedido 1
	\begin{itemize}
		\item Marmita3: R\$ 14,00
		\item Marmita4: R\$ 10,00
		\item SubTotal do pedido: R\$ 24,00
		\item Taxa de entrega: R\$ 4,50
		\item Valor Total do pedido: R\$ 28,50
	\end{itemize}
\end{itemize}

\section{Atores}

\begin{enumerate}
	\item Atendente da empresa, Casa das Marmitas.
\end{enumerate}

\section{Pré-condições}

\begin{enumerate}
	\item O atendente deverá possuir login e senha de acesso autenticados pelo sistema.
	\item O atendente deverá ter permissão para cadastrar pedidos para o cliente.
	\item O atendente deverá ter executado, anteriormente, o caso de uso \nameref{uc013}.
	\item O atendente deverá ter acionado a opção \textbf{Registrar Pedido} da tela \textbf{Exibir Cliente}.
\end{enumerate}

\section{Fluxo de eventos}

\subsection{Fluxo básico}

\begin{enumerate}[label=P\arabic*]
	\item O sistema apresenta a tela \textbf{Registrar Pedido} com os campos \ref{uc007_rn:1}. \label{uc007_p:1}
	\item No campo \textbf{Produto}, o atendente informa o código ou nome de uma marmita. \label{uc007_p:2}
	\item O sistema carrega o caso de uso \nameref{uc012} de acordo com o dado informado no passo anterior.
	\item O atendente seleciona a marmita. \label{uc007_p:4}
	\item O atendente preenche os outros campos da tela. \label{uc007_p:5}	
	\item O atendente aciona a opção \textbf{Incluir item}.
	\item O sistema valida os dados dos campos. \ref{uc007_e:1} \ref{uc007_e:2} \ref{uc007_e:3}
	\item O sistema atualiza a lista de itens de pedido.
	\item O sistema atualiza o valor total do pedido.
	\item O sistema limpa o formulário, com exceção do campo \textbf{Total do pedido}, preenchido pelo atendente.
	\item O atendente aciona a opção \textbf{Salvar}. \label{uc007_a:11}\ref{uc007_a:1} \ref{uc007_a:2} \ref{uc007_a:3}
	\item O sistema registra, com status \textbf{PENDENTE}, o pedido.
	\item O sistema atualiza a lista de pedidos do cliente na tela \textbf{Exibir Cliente}.
	\item O sistema fecha a tela atual.
	\item O sistema retorna ao caso de uso \nameref{uc013}.
	\item Esse caso de uso é encerrado.	
\end{enumerate}

\subsection{Fluxos alternativos}

\begin{enumerate}[label=A\arabic*]
	\item Alternativa ao passo \ref{uc007_a:11} - Incluir novo item de pedido \label{uc007_a:1}
	\begin{enumerate}[label*=.\arabic*]
		\item O atendente volta ao passo \ref{uc007_p:2}.
	\end{enumerate}
	
	\item Alternativa ao passo \ref{uc007_a:11} - Calcular Troco \label{uc007_a:2}
	\begin{enumerate}[label*=.\arabic*]
		\item O atendente aciona a opção \textbf{Calcular Troco}.
		\item O sistema executa o caso de uso \nameref{uc008}.
		\item Esse caso de uso é suspenso.
	\end{enumerate} 
		
	\item Alternativa ao passo \ref{uc007_a:11} - Cancelar inclusão \label{uc007_a:3}
	\begin{enumerate}[label*=.\arabic*]
		\item O atendente aciona a opção \textbf{Cancelar}.
		\item O sistema exibe a mensagem \textbf{Operação cancelada}.
		\item O atendente aciona a opção \textbf{Ok}.
		\item O sistema fecha a tela atual.
		\item O sistema retorna ao caso de uso \nameref{uc013}.
		\item Esse caso de uso é encerrado.
	\end{enumerate} 	
\end{enumerate}

\subsection{Exceções}

\begin{enumerate}[label=E\arabic*]
	\item O atendente não selecionou a marmita \label{uc007_e:1}
	\begin{enumerate}[label*=.\arabic*]
		\item[] No passo \ref{uc007_p:4}, o atendente não selecionou a marmita.
		\item O sistema exibe a mensagem \textbf{Informe a marmita}.
		\item O sistema destaca o campo \textbf{Produto}.
		\item O sistema retorna ao passo \ref{uc007_p:2}.
	\end{enumerate}
	
	\item O atendente não informou algum campo obrigatório \label{uc007_e:2}
	\begin{enumerate}[label*=.\arabic*]
		\item[] No passo \ref{uc007_p:5}, o atendente deixou em branco pelo menos um campo obrigatório.
		\item O sistema exibe a mensagem \textbf{Favor preencher o campo obrigatório}.
		\item O sistema destaca os campos não informados pelo gerente.
		\item O sistema retorna ao passo anterior.
	\end{enumerate}
	
	\item O atendente preencheu de forma errada o campo de quantidade \label{uc007_e:3}
	\begin{enumerate}[label*=.\arabic*]		
		\item[] No passo \ref{uc007_p:5}, o atendente não preencheu de forma correta o campo de quantidade.		
		\item O sistema exibe a mensagem \textbf{Quantidade informada não é válida}.
		\item O sistema destaca o campo \textbf{Quantidade}.
		\item O sistema retorna ao passo anterior.
	\end{enumerate}
\end{enumerate}

\section{Pós-condições}

\begin{enumerate}
	\item O atendente terá cadastrado um novo pedido para o cliente.
	\item O sistema terá retornado ao caso de uso \nameref{uc013}.	
\end{enumerate}

\section{Regras de negócios especiais}

\begin{enumerate}[label=RN\arabic*]
	\item Exibe os campos de dados do item de pedido de acordo com a tabela \ref{uc007_tb_rn1}. \label{uc007_rn:1}
	\begin{table}[htb]
		\ABNTEXfontereduzida
		\caption[Campos de dados do item de pedido]{Campos de dados do item de pedido.}
		\label{uc007_tb_rn1}
		\begin{tabular}{|p{3.0cm}|p{2.0cm}|p{1.5cm}|p{2.0cm}|p{5.75cm}|}
			\hline
			\textbf{Campo}  & \textbf{Tipo} & \textbf{Tamanho} & \textbf{Obrigatório} & \textbf{Observação}                                           \\ \hline
			Produto         & String        & 60               & SIM                  & N.A                                                           \\ \hline
			Tamanho         & Enum          & N.A              & SIM                  & N.A                                                           \\ \hline
			Quantidade      & Short         & N.A              & SIM                  & N.A                                                           \\ \hline
			Total do pedido & Float         & N.A              & SIM                  & O sistema exibirá, no formato moeda, o valor total do pedido. \\ \hline
		\end{tabular}
	\end{table}
\end{enumerate}
\chapter{Calcular Troco - UC008} \label{uc008}

\section{Breve descrição}

Quando o pagamento for em dinheiro, o atendente calculará o troco do cliente.

\section{Atores}

\begin{enumerate}
	\item Atendente da empresa, Casa das Quentinhas.
\end{enumerate}

\section{Pré-condições}

\begin{enumerate}
	\item O atendente deverá possuir login e senha de acesso autenticados pelo sistema.
	\item O atendente deverá ter permissão para realizar o pedido do cliente.
	\item O atendente deverá ter executado, anteriormente, o caso de uso \nameref{uc007}.
	\item O atendente deverá ter acionado a opção \textbf{Calcular Troco} da tela \textbf{Registrar Pedido}.
\end{enumerate}

\section{Fluxo de eventos}

\subsection{Fluxo básico}

\begin{enumerate}[label=P\arabic*]
	\item O sistema apresenta a tela \textbf{Calcular Troco} com os campos \ref{uc008_rn:1}. \label{uc008_p:1}
	\item No campo \textbf{Quantia recebida}, o atendente informa a quantia recebida em dinheiro. \label{uc008_p:2}	
	\item O atendente aciona a opção \textbf{Calcular}. \label{uc008_p:3}\ref{uc008_a:1}
	\item O sistema preenche o campo \textbf{Troco} com valor calculado. \ref{uc008_e:1}
	\item O atendente visualiza o valor calculado.
	\item O atendente vai para o passo \ref{uc008_a:1}.	
\end{enumerate}

\subsection{Fluxos alternativos}

\begin{enumerate}[label=A\arabic*]
	\item Alternativa ao passo \ref{uc008_p:3} - Cancelar cálculo \label{uc008_a:1}
	\begin{enumerate}[label*=.\arabic*]
		\item O atendente aciona a opção \textbf{Cancelar}.
		\item O sistema fecha a tela atual.
		\item O sistema retorna ao caso de uso \nameref{uc007}.
		\item Esse caso de uso é encerrado.
	\end{enumerate} 	
\end{enumerate}

\subsection{Exceções}

\begin{enumerate}[label=E\arabic*]
	\item O atendente não informou o campo de quantia recebida \label{uc008_e:1}
	\begin{enumerate}[label*=.\arabic*]
		\item[] No passo \ref{uc008_p:2}, o atendente deixou em branco o campo de quantia recebida.
		\item O sistema exibe a mensagem \textbf{Favor preencher o campo obrigatório}.
		\item O sistema destaca os campos \textbf{Quantia recebida}.
		\item O sistema retorna ao passo anterior.
	\end{enumerate}
	
	\item O atendente preencheu de forma errada o campo de quantia recebida \label{uc008_e:2}
	\begin{enumerate}[label*=.\arabic*]		
		\item[] No passo \ref{uc008_p:2}, o atendente não preencheu de forma correta o campo de quantia recebida.		
		\item O sistema exibe a mensagem \textbf{A quantia recebida informada não é válida}.
		\item O sistema destaca o campo \textbf{Quantia recebida}.
		\item O sistema retorna ao passo anterior.
	\end{enumerate}
\end{enumerate}

\section{Pós-condições}

\begin{enumerate}
	\item O atendente terá calculado o troco do cliente.
	\item O sistema terá retornado ao caso de uso \nameref{uc007}.	
\end{enumerate}

\section{Regras de negócios especiais}

\begin{enumerate}[label=RN\arabic*]
	\item Exibe os campos referente à tela \textbf{Calcular Troco} de acordo com a tabela \ref{uc008_tb_rn1}. \label{uc008_rn:1}
	\begin{table}[htb]
		\ABNTEXfontereduzida
		\caption[Campos referente à tela \textbf{Calcular Troco}]{Campos referente à tela \textbf{Calcular Troco}.}
		\label{uc008_tb_rn1}
		\begin{tabular}{|p{3.0cm}|p{2.0cm}|p{1.5cm}|p{2.0cm}|p{5.75cm}|}
			\hline
			\textbf{Campo}   & \textbf{Tipo} & \textbf{Tamanho} & \textbf{Obrigatório} & \textbf{Observação}                                                              \\ \hline
			Total do pedido  & Float         & N.A              & SIM                  & O sistema exibirá, no formato moeda, o valor total do pedido.                    \\ \hline
			Quantia recebida & Float         & N.A              & SIM                  & O atendente terá que informar, no formato moeda, a quantia recebida em dinheiro. \\ \hline
			Troco            & Float         & N.A              & SIM                  & O sistema exibirá, no formato moeda, o o troco.                                  \\ \hline
		\end{tabular}
	\end{table}
\end{enumerate}
\chapter{Realizar Entrega - UC009} \label{uc009}

\section{Breve descrição}

Os pedidos serão atendidos com base nas ligações recebidas e o atendimento ao pedido deverá respeitar a ordem das ligações. Os pedidos ficarão nos seguintes status:

\begin{itemize}
	\item Pendente: Quando o atendente da entrada no pedido.
	\item Em trânsito: Quando o motoboy retira o pedido para a entrega.
	\item Cancelado: Quando surge alguma irregularidade e o pedido não pode ser entregue.
	\item Entregue: Quando o motoboy conclui a entrega e apresenta o pagamento.
\end{itemize}

\section{Atores}

\begin{enumerate}
	\item Atendente da empresa, Casa das Quentinhas.
\end{enumerate}

\section{Pré-condições}

\begin{enumerate}
	\item O atendente deverá possuir login e senha de acesso autenticados pelo sistema.
	\item O atendente deverá ter permissão para alterar o status do pedido do cliente.
	\item O atendente deverá ter executado, anteriormente, o caso de uso \nameref{uc013}.
	\item O atendente deverá ter selecionado o pedido na lista de pedidos da tela \textbf{Exibir Cliente}.
\end{enumerate}

\section{Fluxo de eventos}

\subsection{Fluxo básico}

\begin{enumerate}[label=P\arabic*]
	\item O sistema apresenta a tela \textbf{Realizar Entrega} com os campos \ref{uc009_rn:1}. \label{uc009_p:1}
	\item No campo \textbf{Status}, o atendente informa a situação do pedido. \label{uc009_p:2}		
	\item O atendente aciona a opção \textbf{Salvar}. \label{uc009_p:3}\ref{uc009_a:1}
	\item O sistema valida os dados dos campos. 
	\item O sistema grava a alteração.
	\item O sistema atualiza a lista de pedidos do cliente na tela \textbf{Exibir Cliente}.
	\item O sistema fecha a tela atual.
	\item O sistema retorna ao caso de uso \nameref{uc013}.
	\item Esse caso de uso é encerrado.	
\end{enumerate}

\subsection{Fluxos alternativos}

\begin{enumerate}[label=A\arabic*]
	\item Alternativa ao passo \ref{uc009_p:3} - Cancelar alteração \label{uc009_a:1}
	\begin{enumerate}[label*=.\arabic*]
		\item O atendente aciona a opção \textbf{Cancelar}.
		\item O sistema fecha a tela atual.
		\item O sistema retorna ao caso de uso \nameref{uc013}.
		\item Esse caso de uso é encerrado.
	\end{enumerate} 
\end{enumerate}

\section{Pós-condições}

\begin{enumerate}
	\item O atendente terá atualizado o status do pedido do cliente.
	\item O sistema terá retornado ao caso de uso \nameref{uc013}.	
\end{enumerate}

\section{Regras de negócios especiais}

\begin{enumerate}[label=RN\arabic*]
	\item Exibe os campos de dados do pedido de acordo com a tabela \ref{uc009_tb_rn1}. \label{uc009_rn:1}
	\begin{table}[htb]
		\ABNTEXfontereduzida
		\caption[Campos de dados do pedido]{Campos de dados do pedido.}
		\label{uc009_tb_rn1}
		\begin{tabular}{|p{3.0cm}|p{2.0cm}|p{1.5cm}|p{2.0cm}|p{5.75cm}|}
			\hline
			\textbf{Campo}   & \textbf{Tipo} & \textbf{Tamanho} & \textbf{Obrigatório} & \textbf{Observação}                                                                                                                                              \\ \hline
			Cód. do pedido   & Long          & N.A & SIM & N.A                                                                                                                                                              \\ \hline
			Data cadastro    & Date          & N.A & SIM & O sistema exibirá a data de cadastro no formato \textbf{dia/mês/ano hora:minuto}.                                                                                \\ \hline
			Quantidade total & Short         & N.A & SIM & N.A                                                                                                                                                              \\ \hline
			Total do pedido  & Float         & N.A & SIM & O sistema exibirá, no formato moeda, o valor total do pedido.                                                                                                    \\ \hline
			Status           & Enum          & N.A & SIM & O atendente terá que informar uma das seguintes opções: 	
			\begin{enumerate}
				\item PENDENTE (pedido pendente);
				\item TRANSITO (pedido em trânsito);
				\item CANCELADO (pedido cancelado);
				\item ENTREGUE (pedido entregue).
			\end{enumerate}\\ \hline
		\end{tabular}
	\end{table}
\end{enumerate}
\chapter{Gerar Relatório Semanal - UC010} \label{uc010}

\section{Breve descrição}

O Sr. Paulo Ricardo informou também que semanalmente, seu gerente, deverá gerar pelo sistema, um relatório das Entregas Realizadas para o pagamento dos entregadores.

\section{Atores}

\begin{enumerate}
	\item Gerente da empresa, Casa das Quentinhas.
\end{enumerate}

\section{Pré-condições}

\begin{enumerate}
	\item O gerente deverá possuir login e senha de acesso autenticados pelo sistema.
	\item O gerente deverá ter permissão para gerar relatório semanal.
\end{enumerate}

\section{Fluxo de eventos}

\subsection{Fluxo básico}

\begin{enumerate}[label=P\arabic*]
	\item O gerente aciona a opção \textbf{Relatório >> Gerar} no menu do sistema. \label{uc010_p:1}
	\item O sistema apresenta a tela \textbf{Gerar Relatório} com os campos \ref{uc010_rn:1}.	
	\item O gerente preenche os campos da tela. \label{uc010_p:3}	
	\item O gerente aciona a opção \textbf{Gerar Relatório}. \label{uc010_p:4}\ref{uc010_a:1}	
	\item O sistema gera o relatório.
	\item Esse caso de uso é encerrado.	
\end{enumerate}

\subsection{Fluxos alternativos}

\begin{enumerate}[label=A\arabic*]	
	\item Alternativa ao passo \ref{uc010_p:4} - Cancelar relatório \label{uc010_a:1}
	\begin{enumerate}[label*=.\arabic*]
		\item O gerente aciona a opção \textbf{Cancelar}.
		\item O sistema exibe a mensagem \textbf{Operação cancelada}.
		\item O funcionário aciona a opção \textbf{Ok}.
		\item O sistema retorna à tela principal.
		\item Esse caso de uso é encerrado.
	\end{enumerate}			 	
\end{enumerate}

\section{Pós-condições}

\begin{enumerate}
	\item O sistema exibirá o relatório semanal.	
\end{enumerate}

\section{Regras de negócios especiais}

\begin{enumerate}[label=RN\arabic*]
	\item Exibe os campos referente à tela \textbf{Gerar Relatório} de acordo com a tabela \ref{uc010_tb_rn1}. \label{uc010_rn:1}
	\begin{table}[htb]
		\ABNTEXfontereduzida
		\caption[Campos referente à tela \textbf{Gerar Relatório}]{Campos referente à tela \textbf{Gerar Relatório}}
		\label{uc010_tb_rn1}
		\begin{tabular}{|p{3.0cm}|p{2.0cm}|p{1.5cm}|p{2.0cm}|p{5.75cm}|}
			\hline
			\textbf{Campo} & \textbf{Tipo} & \textbf{Tamanho} & \textbf{Obrigatório} & \textbf{Observação}                                                         \\ \hline
			Data inicial   & Date          & N.A              & SIM                  & O gerente terá que informar a data inicial no formato \textbf{dia/mês/ano}. \\ \hline
			Data final     & Date          & N.A              & SIM                  & O gerente terá que informar a data final no formato \textbf{dia/mês/ano}.   \\ \hline
		\end{tabular}
	\end{table}
\end{enumerate}
 
% ----------------------------------------------------------
% ELEMENTOS PÓS-TEXTUAIS
% ----------------------------------------------------------
\postextual
 
 
% ----------------------------------------------------------
% Referências bibliográficas
% ----------------------------------------------------------
%\bibliography{referencias}
 
% ----------------------------------------------------------
% Glossário
% ----------------------------------------------------------

% ---
% Define nome e preâmbulo do glossário
% ---
%\phantompart
%\renewcommand{\glossaryname}{Glossário}
%\renewcommand{\glossarypreamble}{Esta é a descrição do glossário. Experimente
%visualizar outros estilos de glossários, como o \texttt{altlisthypergroup},
%por exemplo.\\
%\\}

% ---
% Traduções para o ambiente glossaries
% ---
%\providetranslation{Glossary}{Glossário}
%\providetranslation{Acronyms}{Siglas}
%\providetranslation{Notation (glossaries)}{Notação}
%\providetranslation{Description (glossaries)}{Descrição}
%\providetranslation{Symbol (glossaries)}{Símbolo}
%\providetranslation{Page List (glossaries)}{Lista de Páginas}
%\providetranslation{Symbols (glossaries)}{Símbolos}
%\providetranslation{Numbers (glossaries)}{Números} 
% ---

% ---
% Estilo de glossário
% ---
% \setglossarystyle{index}
% \setglossarystyle{altlisthypergroup}
%\setglossarystyle{tree}


% ---
% Imprime o glossário
% ---
%\cleardoublepage
%\phantomsection
%\addcontentsline{toc}{chapter}{\glossaryname}
%\printglossaries
% ---
  
 
% ----------------------------------------------------------
% Apêndices
% ----------------------------------------------------------
 
% ----------------------------------------------------------
% Anexos
% ----------------------------------------------------------
 
%---------------------------------------------------------------------
% INDICE REMISSIVO
%---------------------------------------------------------------------
 
\end{document}
